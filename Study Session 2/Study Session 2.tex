\documentclass[../custom]{flashcards}
\usepackage{enumerate}

\newcommand{\studyArea}{Ethics and Professional Standards in Practice}

\def\labelitemii{$\circ$}
\def\labelitemiii{$\diamond$}
\def\labelitemiv{$\cdot$}

\begin{document}
\cardfrontstyle{headings}
\cardfrontfoot{Study Session 2}

\begin{flashcard}[\studyArea]{Asset Manager Code}
    \begin{enumerate}
        \item Loyalty to Clients
        \item Investment Process and Actions
        \item Trading
        \item Risk Management, Compliance, and Support
        \item Performance and Valuation
        \item Disclosures
    \end{enumerate}
\end{flashcard}

\begin{flashcard}[\studyArea]{Loyalty to Clients}
    \begin{itemize}
        \item Salary arrangements that align the interests of the client and the manager without taking undue risks.
        \item Confidential information policy delineating how client information is collected, used and stored.
        \item Anti-money-laundering policy
        \item Defining a \emph{token} gift and allowing only token gifts in business relationships. Cash is not acceptable. Employees should always notify in writing when accepting a gift.
    \end{itemize}
\end{flashcard}

\begin{flashcard}[\studyArea]{Investment Process and Actions}
    \begin{itemize}
        \item Different levels of service and products available to all clients.
        \item Use third party research provided that it has a verified reasonable basis.
        \item Conduct stress testing when using complex derivatives products.
        \item Disclose and get client agreement on events causing strategy changes.
        \item Each client should have an IPS with risk and return objectives and constraints. This should be reviewed at least annually and should include performance benchmarks.
        \item Investments should be made in the context of the client's total assets.
    \end{itemize}
\end{flashcard}

\begin{flashcard}[\studyArea]{Trading}
    \begin{itemize}
        \item Do not trade or cause others to trade on material nonpublic information. Information walls should be created between different departments to restrict the flow of nonpublic information.
        \item Always place client trades before your own. Require employees to have approval for IPOs or private placements. Create a restricted list of securities.
        \item Only use soft dollars to directly aid in the investment decision process. Managers should disclose soft dollar arrangements and how often they are used.
        \item If the client uses a particular broker, they should be notified that they may not receive the best execution.
        \item Allocate shares equitably among all clients. Use block trades or allocate shares on a pro rata basis. Policies for IPOs should be explicitly stated.
    \end{itemize}
\end{flashcard}

\begin{flashcard}[\studyArea]{Risk Management, Compliance, Support}
    \begin{itemize}
        \item Develop polices for complying with the Code and Standards and regulations. Have documentation, internal controls and self-assessment mechanisms.
        \item Establish a firm-wide system for measuring risk.
        \item Appoint a compliance officer who reports to the CEO and is responsible for monitoring and investigating compliance breaches. The officer should distribute the Code to employees and monitor trading activity.
        \item Ensure portfolio information is accurate, complete, and verified.
        \item Maintain records for at least seven years.
        \item Have enough staff and resources to make informed decisions.
        \item Have a BCP in the event of a disaster. The plan should include off-site backup, continuing operations and continuing communications.
    \end{itemize}
\end{flashcard}

\begin{flashcard}[\studyArea]{Performance and Valuation}
    \begin{itemize}
        \item Fairly and accurately report investment results. Do no misrepresent by including accounts that weren't managed or only reporting certain time periods.
        \item Hypothetical models should be fully disclosed.
        \item Valuing assets accounts should be done by a third party to prevent performance misrepresentation.
        \item Client accounts should be valued consistently with accepted techniques.
    \end{itemize}
\end{flashcard}

\begin{flashcard}[\studyArea]{Disclosures}
    \begin{itemize}
        \item Any information needed to make a decision about the manager, organization or process.
        \item Soft or bundled commissions, referral fees, brokerage arrangements, and stocks held by employees.
        \item Any regulatory or disciplinary action taken against the manager or its personnel.
        \item Decision-making process including inherent risks and the extent of derivatives.
        \item Fee schedules, projects and methodologies used in charging fees.
        \item Performance of the client's account on a regular basis.
        \item Valuation methods used to make investment decisions.
        \item Proxy voting policies of the manager.
        \item How trades are allocated.
        \item Results of audits on the client's account.
        \item Significant personnel or organizational changes.
        \item Firm-wide risk management process.
    \end{itemize}
\end{flashcard}

\end{document}
