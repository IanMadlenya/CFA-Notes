\documentclass[../custom,grid]{flashcards}
\usepackage{enumitem}
%\usepackage{amsmath}
%\usepackage{booktabs}
%\usepackage{array}
%\usepackage{multirow}

\newcommand{\studyArea}{Private Wealth Management}

\def\labelitemii{$\circ$}
\def\labelitemiii{$\diamond$}
\def\labelitemiv{$\cdot$}

\begin{document}
\cardfrontstyle{headings}
\cardfrontfoot{Study Session 4}

\begin{flashcard}[\studyArea]{Active and Passive Wealth Creation}
    \begin{itemize}
        \item \textbf{Active wealth creation.} Wealth that has been accumulated through entrepreneurial activity may be the result of risk taking. Thus, this individual could have willingness to take risk. However, they may treat business risk different from investment risk.
        \item \textbf{Passive wealth creation.} Wealth acquired through windfall or inheritance could indicate a lack of knowledge with investment decisions. Thus, this individual may have below-average willingness to tolerate risk.
    \end{itemize}
\end{flashcard}

\begin{flashcard}[\studyArea]{Stages of Life}
    \begin{itemize}
        \item \textbf{Foundation} is seeking to accumulate wealth through a job and savings, seeking education, or building a business. Long time horizon can increase risk tolerance. But often have little wealth to risk which may reduce ability.
        \item \textbf{Accumulation} is when earnings or business success rise and assets can be accumulated. Demands such a house or kids' college may rise. Could be a time of maximum savings and wealth accumulation with a higher ability to bear risk.
        \item \textbf{Maintenance}phase often means retirement. Preserving wealth and living off the portfolio are important. Ability to bear risk is declining, but not low. Life expectancy can be long and being too conservative can decrease standard of living.
        \item \textbf{Distribution} stage means assets exceed needs and a process of distributing them to others can start. Might involve gifts or making plans for death. Objectives may extend beyond death so time horizon remains long and ability to bear risk could remain high.
    \end{itemize}
\end{flashcard}

\begin{flashcard}[nosep]{Personality Type Classifications}
    \begin{itemize}[nosep]
        \item \textbf{Cautious investors}
            \begin{itemize}[nosep]
                \item Have strong desire for security.
                \item Prefer safe, low-volatility investments with little potential for loss. 
                \item Do not like making their own investment decisions but are difficult to advise.
                \item Portfolios have low turnover.
            \end{itemize}
        \item \textbf{Methodical investors}
            \begin{itemize}[nosep]
                \item Diligently research markets, industries, and firms to gather information.
                \item Decisions tend to be conservative.
                \item Rarely form emotional attachments to investments.
                \item Seek confirmation of decisions and constantly on the lookout for better information.
            \end{itemize}
        \item \textbf{Individualistic investors}
            \begin{itemize}[nosep]
                \item Do their own research and are confident in their ability to make decisions.
                \item When faced with contradictory information, will devote time to reconcile.
                \item Have confidence in their ability to achieve long-term investment objectives.
            \end{itemize}
        \item \textbf{Spontaneous investors}
            \begin{itemize}[nosep]
                \item Constantly adjust portfolios in response to changing market conditions.
                \item Fear failing to respond to changing market conditions will negatively impact portfolios.
                \item Acknowledge lack of investment expertise, but also doubt advice.
                \item Have high turnover and trading costs.
            \end{itemize}
    \end{itemize}
\end{flashcard}

\begin{flashcard}[\studyArea]{Benefits of the IPS}
    \begin{itemize}
        \item For the client
            \begin{itemize}
                \item Identifies and documents objectives and constraints.
                \item Dynamic, allowing changes in response to changing circumstances or market conditions.
                \item Easily understood, giving the ability to bring in new managers without disruption.
                \item Developing helps clients learn more about themselves and investment decision making and are better able to understand investment recommendations.
            \end{itemize}
        \item For the advisor
            \begin{itemize}
                \item Greater knowledge of the client.
                \item Guidance for investment decision making.
                \item Guidance for resolution of disputes.
            \end{itemize}
    \end{itemize}
\end{flashcard}

\begin{flashcard}[\studyArea]{Return Objective}
    \begin{flushleft}
        Often can be divided in to desired and required components. Required is what is necessary to meet critical goals. Might be living expenses, education or healthcare. Desired might be buying a second home or travel.\newline

        Some managers distinguish between income and growth sources, but this is suboptimal to a total return approach. As long as sufficient return is earned over the long run, funds can be available to meet needs.\newline

        Return objective should specify whether it is nominal (including inflation) or real, and pretax or after-tax.
    \end{flushleft}
\end{flashcard}

\begin{flashcard}[\studyArea]{Risk Objective}
    \begin{itemize}
        \item \textbf{Ability to take risk} is the ability to sustain losses with jeopardizing goals. How much volatility the portfolio can withstand and still meet expenditures. Significantly affected by time horizon and relative size of expenditures.
            \begin{itemize}
                \item As time horizon increases, ability to take risk increases.
                \item Large relative expenditures, reduce ability to take risk.
                \item As the importance of an expense increases, ability to take risk decreases.
                \item If a goal or amount can be changed, the client has flexibility, increasing ability.
                \item If still working or has other assets, this increases the ability.
                \item Liquidity needs can reduce ability.
            \end{itemize}
        \item \textbf{Willingness to take risk} is subjective and determined by analysis of a psychological profile. Rather than accept the client's statement, you should look for confirming or contradicting evidence.
    \end{itemize}
\end{flashcard}

\begin{flashcard}[\studyArea]{Time Horizon Constraint}
    \begin{flushleft}
        Important because it affects ability to bear risk. Most basically, it is the expected remaining years of life. Total number of years the portfolio will be managed to meet the investor's needs. Fifteen years or more is long-term, and short-term is three years or less.\newline

        Many horizons are multistage. A stage is indicated by changes in circumstances or objectives significant enough to require evaluating the IPS and reallocating the portfolio.
    \end{flushleft}
\end{flashcard}

\begin{flashcard}[\studyArea]{Tax Constraint}
    \begin{itemize}
        \item \textbf{Tax defferal.} Minimize potentially compounding effect of taxes by paying them at the end of the investment period. Strategies focus on long-term capital gains, low turnover, and loss harvesting.
        \item \textbf{Tax avoidance.} Invest in tax-free securities. Special savings accounts and tax-free municipal bonds are examples of securities with tax-free returns.
        \item \textbf{Tax reduction.} Invest in securities that require less direct tax payment. Capital gains may be taxed a lower rate than income. Annual taxes should be reduced through loss harvesting, when available.
        \item \textbf{Wealth transfer taxes.} Minimize transfer taxes by planning the transfer without utilizing a sale. Often quite specific to the jurisdiction. Considering the timing of the transfer is also important.
    \end{itemize}
\end{flashcard}

\begin{flashcard}[\studyArea]{Liquidity Constraint}
    \begin{itemize}
        \item Ongoing needs for distributions such as living expenses.
        \item Emergency reserves for unanticipated distributions if agreed to in advance. Otherwise they create cash drag.
        \item One-time or infrequent negative liquidity events requiring irregular distributions should be noted.
        \item Positive liquidity events not due to assets should also be noted.
        \item Illiquid assets, restricted from sale or causing a large tax bill on sale, should be noted.
        \item Ownership of a home is generally an illiquid asset and could be noted.
    \end{itemize}
\end{flashcard}

\begin{flashcard}[\studyArea]{Legal and Regulatory Constraint}
    \begin{flushleft}
        Trusts are formed legal devices for transferring personal wealth to future generations. In a revocable trust, the grantor retains ownership and control over the assets and is responsible for taxes. Often manages the assets personally or hires a manager.\newline

        In an irrevocable trust, the grantor confers ownership of the assets to the trust. The assets are considered immediately transferred and can be subject to wealth transfer taxes. The trust is a taxable entity and will file tax returns and pay any taxes due. The original grantor no longer has control of the assets and is not taxed on them.\newline

        Family foundations are another vehicle, similar to an irrevocable trust, used to transfer assets to future generations. Family members frequently remain as managers of the foundation's assets.
    \end{flushleft}
\end{flashcard}

\begin{flashcard}[\studyArea]{Unique Circumstances Constraint}
    \begin{itemize}
        \item Special investment concerns (e.g., socially responsible investing).
        \item Special instructions (e.g., gradually liquidate a holding over a period of time).
        \item Restrictions on the sale of assets (e.g., a large holding of a single stock).
        \item Asset classes the client specifically forbids or limits (e.g., position limits on asset classes or totally disallowed asset classes).
        \item Assets held outside the investable portfolio (e.g., a primary or secondary residence).
        \item Desired bequests (e.g., the client intends to leave his home or a given amount of wealth to children or charity).
        \item Desired objectives not attainable due to time horizon or current wealth.
    \end{itemize}
\end{flashcard}

\begin{flashcard}[\studyArea]{Monte Carlo Approach to Retirement Planning}
    \begin{itemize}
        \item Advantages
            \begin{itemize}
                \item Considers path dependency.
                \item More clearly displays tradeoffs of risk and return by ranking paths.
                \item Properly models tax analysis, which considers actual tax rates as well as account types (taxable or tax-deferred).
                \item Clearer understanding of short-term and long-term risk.
                \item Superior in assessing multi-period effects. Can model the stochastic process where return over time depends on the starting value of the period as well as additions and withdrawals.
            \end{itemize}
        \item Disadvantages
            \begin{itemize}
                \item Simplistic use of historical data for inputs. Returns change and have major effects on projected future values of the portfolio.
                \item Models that simulate the return of asset classes but not the actual assets held.
                \item Tax modeling that is simplistic and not tailored to the investor's situation.
            \end{itemize}
    \end{itemize}
\end{flashcard}
\end{document}
