\documentclass[../custom,grid]{flashcards}
\usepackage{enumitem}
\usepackage{booktabs}
\usepackage{array}
\usepackage{amsmath}
\usepackage{tikz}
\usepackage{float}

\newcommand{\studyArea}{Private Wealth Management}

\def\labelitemii{$\circ$}
\def\labelitemiii{$\diamond$}
\def\labelitemiv{$\cdot$}

\begin{document}
\cardfrontstyle{headings}
\cardfrontfoot{Study Session 4}

\begin{flashcard}[\studyArea]{Active and Passive Wealth Creation}
    \begin{itemize}
        \item \textbf{Active wealth creation.} Wealth that has been accumulated through entrepreneurial activity may be the result of risk taking. Thus, this individual could have willingness to take risk. However, they may treat business risk different from investment risk.
        \item \textbf{Passive wealth creation.} Wealth acquired through windfall or inheritance could indicate a lack of knowledge with investment decisions. Thus, this individual may have below-average willingness to tolerate risk.
    \end{itemize}
\end{flashcard}

\begin{flashcard}[\studyArea]{Stages of Life}
    \begin{itemize}
        \item \textbf{Foundation} is seeking to accumulate wealth through a job and savings, seeking education, or building a business. Long time horizon can increase risk tolerance. But often have little wealth to risk which may reduce ability.
        \item \textbf{Accumulation} is when earnings or business success rise and assets can be accumulated. Demands such a house or kids' college may rise. Could be a time of maximum savings and wealth accumulation with a higher ability to bear risk.
        \item \textbf{Maintenance}phase often means retirement. Preserving wealth and living off the portfolio are important. Ability to bear risk is declining, but not low. Life expectancy can be long and being too conservative can decrease standard of living.
        \item \textbf{Distribution} stage means assets exceed needs and a process of distributing them to others can start. Might involve gifts or making plans for death. Objectives may extend beyond death so time horizon remains long and ability to bear risk could remain high.
    \end{itemize}
\end{flashcard}

\begin{flashcard}[nosep]{Personality Type Classifications}
    \begin{itemize}[nosep]
        \item \textbf{Cautious investors}
            \begin{itemize}[nosep]
                \item Have strong desire for security.
                \item Prefer safe, low-volatility investments with little potential for loss. 
                \item Do not like making their own investment decisions but are difficult to advise.
                \item Portfolios have low turnover.
            \end{itemize}
        \item \textbf{Methodical investors}
            \begin{itemize}[nosep]
                \item Diligently research markets, industries, and firms to gather information.
                \item Decisions tend to be conservative.
                \item Rarely form emotional attachments to investments.
                \item Seek confirmation of decisions and constantly on the lookout for better information.
            \end{itemize}
        \item \textbf{Individualistic investors}
            \begin{itemize}[nosep]
                \item Do their own research and are confident in their ability to make decisions.
                \item When faced with contradictory information, will devote time to reconcile.
                \item Have confidence in their ability to achieve long-term investment objectives.
            \end{itemize}
        \item \textbf{Spontaneous investors}
            \begin{itemize}[nosep]
                \item Constantly adjust portfolios in response to changing market conditions.
                \item Fear failing to respond to changing market conditions will negatively impact portfolios.
                \item Acknowledge lack of investment expertise, but also doubt advice.
                \item Have high turnover and trading costs.
            \end{itemize}
    \end{itemize}
\end{flashcard}

\begin{flashcard}[\studyArea]{Benefits of the IPS}
    \begin{itemize}
        \item For the client
            \begin{itemize}
                \item Identifies and documents objectives and constraints.
                \item Dynamic, allowing changes in response to changing circumstances or market conditions.
                \item Easily understood, giving the ability to bring in new managers without disruption.
                \item Developing helps clients learn more about themselves and investment decision making and are better able to understand investment recommendations.
            \end{itemize}
        \item For the advisor
            \begin{itemize}
                \item Greater knowledge of the client.
                \item Guidance for investment decision making.
                \item Guidance for resolution of disputes.
            \end{itemize}
    \end{itemize}
\end{flashcard}

\begin{flashcard}[\studyArea]{Return Objective}
    \begin{flushleft}
        Often can be divided in to desired and required components. Required is what is necessary to meet critical goals. Might be living expenses, education or healthcare. Desired might be buying a second home or travel.\newline

        Some managers distinguish between income and growth sources, but this is suboptimal to a total return approach. As long as sufficient return is earned over the long run, funds can be available to meet needs.\newline

        Return objective should specify whether it is nominal (including inflation) or real, and pretax or after-tax.
    \end{flushleft}
\end{flashcard}

\begin{flashcard}[\studyArea]{Risk Objective}
    \begin{itemize}
        \item \textbf{Ability to take risk} is the ability to sustain losses with jeopardizing goals. How much volatility the portfolio can withstand and still meet expenditures. Significantly affected by time horizon and relative size of expenditures.
            \begin{itemize}
                \item As time horizon increases, ability to take risk increases.
                \item Large relative expenditures, reduce ability to take risk.
                \item As the importance of an expense increases, ability to take risk decreases.
                \item If a goal or amount can be changed, the client has flexibility, increasing ability.
                \item If still working or has other assets, this increases the ability.
                \item Liquidity needs can reduce ability.
            \end{itemize}
        \item \textbf{Willingness to take risk} is subjective and determined by analysis of a psychological profile. Rather than accept the client's statement, you should look for confirming or contradicting evidence.
    \end{itemize}
\end{flashcard}

\begin{flashcard}[\studyArea]{Time Horizon Constraint}
    \begin{flushleft}
        Important because it affects ability to bear risk. Most basically, it is the expected remaining years of life. Total number of years the portfolio will be managed to meet the investor's needs. Fifteen years or more is long-term, and short-term is three years or less.\newline

        Many horizons are multistage. A stage is indicated by changes in circumstances or objectives significant enough to require evaluating the IPS and reallocating the portfolio.
    \end{flushleft}
\end{flashcard}

\begin{flashcard}[\studyArea]{Tax Constraint}
    \begin{itemize}
        \item \textbf{Tax defferal.} Minimize potentially compounding effect of taxes by paying them at the end of the investment period. Strategies focus on long-term capital gains, low turnover, and loss harvesting.
        \item \textbf{Tax avoidance.} Invest in tax-free securities. Special savings accounts and tax-free municipal bonds are examples of securities with tax-free returns.
        \item \textbf{Tax reduction.} Invest in securities that require less direct tax payment. Capital gains may be taxed a lower rate than income. Annual taxes should be reduced through loss harvesting, when available.
        \item \textbf{Wealth transfer taxes.} Minimize transfer taxes by planning the transfer without utilizing a sale. Often quite specific to the jurisdiction. Considering the timing of the transfer is also important.
    \end{itemize}
\end{flashcard}

\begin{flashcard}[\studyArea]{Liquidity Constraint}
    \begin{itemize}
        \item Ongoing needs for distributions such as living expenses.
        \item Emergency reserves for unanticipated distributions if agreed to in advance. Otherwise they create cash drag.
        \item One-time or infrequent negative liquidity events requiring irregular distributions should be noted.
        \item Positive liquidity events not due to assets should also be noted.
        \item Illiquid assets, restricted from sale or causing a large tax bill on sale, should be noted.
        \item Ownership of a home is generally an illiquid asset and could be noted.
    \end{itemize}
\end{flashcard}

\begin{flashcard}[\studyArea]{Legal and Regulatory Constraint}
    \begin{flushleft}
        Trusts are formed legal devices for transferring personal wealth to future generations. In a revocable trust, the grantor retains ownership and control over the assets and is responsible for taxes. Often manages the assets personally or hires a manager.\newline

        In an irrevocable trust, the grantor confers ownership of the assets to the trust. The assets are considered immediately transferred and can be subject to wealth transfer taxes. The trust is a taxable entity and will file tax returns and pay any taxes due. The original grantor no longer has control of the assets and is not taxed on them.\newline

        Family foundations are another vehicle, similar to an irrevocable trust, used to transfer assets to future generations. Family members frequently remain as managers of the foundation's assets.
    \end{flushleft}
\end{flashcard}

\begin{flashcard}[\studyArea]{Unique Circumstances Constraint}
    \begin{itemize}
        \item Special investment concerns (e.g., socially responsible investing).
        \item Special instructions (e.g., gradually liquidate a holding over a period of time).
        \item Restrictions on the sale of assets (e.g., a large holding of a single stock).
        \item Asset classes the client specifically forbids or limits (e.g., position limits on asset classes or totally disallowed asset classes).
        \item Assets held outside the investable portfolio (e.g., a primary or secondary residence).
        \item Desired bequests (e.g., the client intends to leave his home or a given amount of wealth to children or charity).
        \item Desired objectives not attainable due to time horizon or current wealth.
    \end{itemize}
\end{flashcard}

\begin{flashcard}[\studyArea]{Monte Carlo Approach to Retirement Planning}
    \begin{itemize}
        \item Advantages
            \begin{itemize}
                \item Considers path dependency.
                \item More clearly displays tradeoffs of risk and return by ranking paths.
                \item Properly models tax analysis, which considers actual tax rates as well as account types (taxable or tax-deferred).
                \item Clearer understanding of short-term and long-term risk.
                \item Superior in assessing multi-period effects. Can model the stochastic process where return over time depends on the starting value of the period as well as additions and withdrawals.
            \end{itemize}
        \item Disadvantages
            \begin{itemize}
                \item Simplistic use of historical data for inputs. Returns change and have major effects on projected future values of the portfolio.
                \item Models that simulate the return of asset classes but not the actual assets held.
                \item Tax modeling that is simplistic and not tailored to the investor's situation.
            \end{itemize}
    \end{itemize}
\end{flashcard}

\begin{flashcard}[\studyArea]{Types of Taxes}
    \begin{itemize}
        \item \textbf{Taxes on income}
            \begin{itemize}
                \item Paid by individuals, corporations, and other legal entities on income including wages, interest, dividends, and capital gains.
            \end{itemize}
        \item \textbf{Wealth-based taxes}
            \begin{itemize}
                \item Paid on the value of assets held and on wealth transferred.
            \end{itemize}
        \item \textbf{Taxes on compensation}
            \begin{itemize}
                \item Sales taxes paid by the consumer.
                \item Value-added taxes paid at each intermediate production step according to value added at the step. Ultimately borne by the consumer.
            \end{itemize}
    \end{itemize}
\end{flashcard}

\begin{flashcard}[\studyArea]{Characteristics of Tax Regimes}
    \begin{tabular}
        {>{\raggedright}m{1.4in}
         >{\raggedright}m{0.7in}
         >{\raggedright}m{0.7in}
         >{\raggedright}m{0.7in}
         >{\raggedright\arraybackslash}m{0.7in}} \toprule

        \textit{Tax Regime} &
        \textit{Ordinary Income Tax Structure} &
        \textit{Favorable Treatment for Interest Income?} &
        \textit{Favorable Treatment for Dividend Income?} &
        \textit{Favorable Treatment for Capital Gains?}\\ \midrule

        Common Progressive &
        Progressive &
        Yes &
        Yes &
        Yes\\ \addlinespace

        Heavy Dividend Tax &
        Progressive &
        Yes &
        No &
        Yes\\ \addlinespace

        Heavy Capital Gain Tax &
        Progressive &
        Yes &
        Yes &
        No\\ \addlinespace

        Heavy Interest Tax &
        Progressive &
        No &
        Yes &
        Yes\\ \addlinespace

        Light Capital Gain Tax &
        Progressive &
        No &
        No &
        Yes\\ \addlinespace

        Flat and Light &
        Flat &
        Yes &
        Yes &
        Yes\\ \addlinespace

        Flat and Heavy &
        Flat &
        Yes &
        No &
        No\\ \bottomrule
    \end{tabular}
\end{flashcard}

\begin{flashcard}[\studyArea]{Accrual Taxes}
    \begin{flushleft}
        \begin{align*}
            \text{FVIF}_{\text{IT}} &= (1 + R(1 - T_{\text{I}}))^N\\
            \\
            \text{where:}\\
            \text{FVIF}_{\text{IT}} &= \text{future value interest factor after investment income tax}\\
            R &= \text{before-tax investment return}\\
            T_{\text{I}} &= \text{annual tax rate on investment income}\\
            N &= \text{number of investment periods}
        \end{align*}

        \begin{itemize}
            \item $\text{TD}_{\%} > T_{\text{I}}$
            \item As time horizon increases, $\text{TD}_{\$}$ and $\text{TD}_{\%}$ increase.
            \item As return increases, $\text{TD}_{\$}$ and $\text{TD}_{\%}$ increase.
        \end{itemize}
    \end{flushleft}
\end{flashcard}

\begin{flashcard}[\studyArea]{Capital Gains Taxes}
    \begin{flushleft}
        \begin{align*}
            \text{FVIF}_{\text{CGT}} &= (1 + R)^N (1 - T_{\text{CG}}) + T_{\text{CG}}B\\
            \\
            \text{where:}\\
            \text{FVIF}_{\text{CGT}} &= \text{future value interest factor after capital gains tax}\\
            R &= \text{before-tax investment return}\\
            T_{\text{CG}} &= \text{tax rate on capital gains}\\
            N &= \text{number of investment periods}\\
            B &= \text{ratio of cost basis to current market value}
        \end{align*}

        \begin{itemize}
            \item $\text{TD}_{\%} = T_{\text{CG}}$
            \item As time horizon increases, $\text{TD}_{\$}$ and $\text{TD}_{\%}$ are unchanged.
            \item As return increases, $\text{TD}_{\$}$ and $\text{TD}_{\%}$ are unchanged.
            \item As time horizon increases, value of the tax deferral increases.
            \item As return increases, value of the tax deferral increases.
        \end{itemize}
    \end{flushleft}
\end{flashcard}

\begin{flashcard}[\studyArea]{Wealth-Based Taxes}
    \begin{flushleft}
        \begin{align*}
            \text{FVIF}_{\text{WT}} &= ((1 + R)(1 - T_{\text{W}}))^N\\
            \\
            \text{where:}\\
            \text{FVIF}_{\text{WT}} &= \text{future value interest factor after wealth-based tax}\\
            R &= \text{before-tax investment return}\\
            T_{\text{W}} &= \text{wealth-based tax rate}\\
            N &= \text{number of investment periods}
        \end{align*}

        \begin{itemize}
            \item $\text{TD}_{\%} > T_{\text{W}}$
            \item As time horizon increases, $\text{TD}_{\$}$ and $\text{TD}_{\%}$ increase.
            \item As return increases, $\text{TD}_{\$}$ increases and $\text{TD}_{\%}$ decreases.
        \end{itemize}
    \end{flushleft}
\end{flashcard}

\begin{flashcard}[\studyArea]{Future Value Interest Factor Considering All Taxes}
    \begin{flushleft}
        Start with annual return after taxes.
        \begin{align*}
            R_{\text{ART}} &= R(1 - \text{realized tax rate})\\
            &= R(1 - (P_{\text{I}}T_{\text{I}} + P_{\text{D}}T_{\text{D}} + P_{\text{CG}}T_{\text{CG}}))\\
            \\
            \text{where:}\\
            R_{\text{ART}} &= \text{annual return after realized taxes}\\
            R &= \text{before-tax investment return}\\
            P_\text{\text{X}} &= \text{proportion of total return from X, taxed at $T_{\text{X}}$}
        \end{align*}

        We can find the effective capital gains tax rate, $T_{\text{ECG}}$.
        \[
            T_{\text{ECG}} = T_{\text{CG}} \frac{1 - (P_{\text{I}} + P_{\text{D}} + P_{\text{CG}})}{1 - (P_{\text{I}}T_{\text{I}} + P_{\text{D}}T_{\text{D}} + P_{\text{CG}}T_{\text{CG}})}
        \]

        Finally, we have the future value interest factor considering all taxes as well as cost basis.
        \[
            \text{FVIF}_{\text{T}} = (1 + R_{\text{ART}})^N (1 - T_{\text{ECG}}) + T_{\text{ECG}} - (1 - B)T_{\text{CG}}
        \]
    \end{flushleft}
\end{flashcard}

\begin{flashcard}[\studyArea]{Accrual Equivalent Tax}
    \begin{flushleft}
        The accrual equivalent tax return is the annual return that produces the same terminal value as the taxable portfolio.
        \[
            R_{\text{AE}} = \left ( \frac{\text{FV}_T}{\text{PV}} \right )^{\frac{1}{N}} - 1 
        \]

        The accrual equivalent tax rate is the tax rate that makes the pre-tax return equal to the accrual equivalent after-tax return.
        \[
            T_{\text{AE}} = 1 - \frac{R_{\text{AE}}}{R}
        \]
    \end{flushleft}
\end{flashcard}

\begin{flashcard}[\studyArea]{Tax-Deferred Account}
    \begin{flushleft}
        Contributions reduce current taxes, and returns on the contributions accrue tax fee. They are taxed when withdrawn from. They have front-end loaded tax benefits.
        \begin{align*}
            \text{FVIF}_{\text{TDA}} &= (1 + R)^N (1 - T_N)\\
            \\
            \text{where:}\\
            \text{FVIF}_{\text{TDA}} &= \text{future value interest factor for a TDA}\\
            R &= \text{before-tax return on the account}\\
            T_N &= \text{tax rate in effect at the time of the withdrawal}
        \end{align*}
    \end{flushleft}
\end{flashcard}

\begin{flashcard}[\studyArea]{Tax-Exempt Account}
    \begin{flushleft}
        Contributions are made with after-tax funds and do not reduce the current tax bill. Funds are withdrawn tax-free, and so they have back-end loaded tax benefits.
        \begin{align*}
            \text{FVIF}_{\text{TEA}} &= (1 + R)^N (1 - T_0)\\
            \\
            \text{where:}\\
            \text{FVIF}_{\text{TEA}} &= \text{future value interest factor for a TDA}\\
            R &= \text{before-tax return on the account}\\
            T_0 &= \text{tax rate in effect at the time of the contribution}
        \end{align*}
    \end{flushleft}
\end{flashcard}

\begin{flashcard}[\studyArea]{After-Tax Investment Risk}
    \begin{flushleft}
        If investment returns are taxed solely as income at the rate of $T_{\text{I}}$ and the pre-tax standard deviation of returns is $\sigma$, the after-tax risk is $\sigma (1 - T_{\text{I}})$.\newline

        If the investment is held in a tax-exempt account, the investor bears all the investment risk. This is also true for TDAs prior to withdrawal because annual returns are not subject to taxes.
    \end{flushleft}
\end{flashcard}

\begin{flashcard}[\studyArea]{Tax Alpha}
    \begin{flushleft}
        Value created by the effective tax management of investments. From a tax-management standpoint, an investor should locate heavily taxed assets in tax-advantaged accounts and lightly taxed assets in taxable accounts.\newline

        In most countries, the strategy would be to put equities in taxable accounts because their current income is lower and capital gains can be deferred. Bonds, with higher current income, would be placed in a tax-protected account.
    \end{flushleft}
\end{flashcard}

\begin{flashcard}[\studyArea]{After-Tax Returns of Investor Types}
    \begin{itemize}
        \item \textbf{Traders.} Due to frequent trading, traders forgo the tax advantages associated with equity. All gains are short-term and taxed annually.
        \item \textbf{Active investors.} Trade less frequently than traders so many gains are longer term and taxed at lower rates.
        \item \textbf{Passive investors.} Buy and hold equity so that gains are deferred long term and taxed at preferential rates.
        \item \textbf{Exempt investors.} Hold all their stock in tax-exempt accounts, thereby avoiding taxation altogether.
    \end{itemize}
\end{flashcard}

\begin{flashcard}[\studyArea]{Tax Loss Harvesting}
    \begin{flushleft}
        The practice of using investment losses to offset investment gains or income and thus avoid the associated taxes. Government may place limits on the amount of losses that can be recognized or the type of gains that can be offset.\newline

        Although it saves current taxes, it generally does not save on cumulative taxes as it usually raises future tax bills.
    \end{flushleft}
\end{flashcard}

\begin{flashcard}[\studyArea]{Highest-In First Out Tax Lot Accounting}
    \begin{flushleft}
        When allowed, investors can generate significant tax savings by using HIFO accounting to liquidate positions with the highest cost basis. Total taxes over time are unchanged with HIFO, but it allows tax savings to be reinvested earlier, creating a tax alpha.\newline

        If tax rates are not expected to be constant, the value of HIFO can vary. If tax rates are expected to rise, it could be beneficial to use LIFO accounting and associate the higher gain with the lower rate.\newline

        Volatile security prices have the most potential for creating tax alpha.\newline

        Though excessive trading can create tax inefficiencies, a limited amount of trading can be beneficial when capital losses can be harvested.
    \end{flushleft}
\end{flashcard}

\begin{flashcard}[\studyArea]{Holding Period Management}
    \begin{itemize}
        \item Patient traders win out over rapid traders.
        \item Rapid trading would require a much higher pre-tax return to break even.
        \item If a sale is being considered near the tax year end, make the sale
            \begin{itemize}
                \item Before year end if it is a loss in order to place the loss in the current tax year and offset gains this year. This will lower taxes this year, but raise taxes next year.
                \item After year end if it is a gain. This will defer the gain and tax until next year's tax return.
            \end{itemize}
        \item If tax rates are going to change, the analysis could become more complicated.
    \end{itemize}
\end{flashcard}

\begin{flashcard}[\studyArea]{}
    \begin{flushleft}
        Ideally, the efficient frontier should be viewed on an after-tax basis. Because the tax status of an investment depends on the account it's in, it could appear on the efficient frontier in both taxable and non-taxable forms.\newline

        The mean-variance optimization should optimally allocate assets and determine the optimal asset location for each asset. Accrual equivalent after-tax returns would be substituted for before-tax returns, and risk on an after-tax basis would be substituted for before-tax risk.
    \end{flushleft}
\end{flashcard}

\begin{flashcard}[\studyArea]{Probate}
    \begin{flushleft}
        A legal process that takes place at death, during which a court determines the validity of the decedent's will, inventories the decedent's property, resolves any claims against the decedent, and distributes remaining property according to the will.\newline

        Involves considerable costs which are borne by the decedent's estate. If the decedent leaves no will, or if the will is deemed invalid, the decedent is said to have died intestate and the distribution of the assets is determined by the court.\newline

        Assets solely owned by the decedent must be transferred by a will through the probate process. This can avoided through joint ownership with rights of survivorship, living trusts, retirement plans, and life insurance.
    \end{flushleft}
\end{flashcard}

\begin{flashcard}[\studyArea]{Gifts and Bequests}
    \begin{itemize}
        \item \textbf{Gifts} are referred to as lifetime gratuitous transfer or inter vivos transfers and may be subject to gift taxes. Whether the gift is taxed and who pays the tax is determined by the taxing authorities.
        \item \textbf{Bequests} are referred to as testamentary gratuitous transfers and can be subject to estate taxes, paid by the grantor, or inheritance taxes, paid by the recipient.
    \end{itemize}
\end{flashcard}

\begin{flashcard}[\studyArea]{Ownership Rights}
    \begin{itemize}
        \item \textbf{Forced heirship} means children have a right to a portion of a parent's estate, regardless of the location of the child vis-\`{a}-vis the parent, their relationship, or the parents' relationship.
        \item \textbf{Claw-back} provisions add gifts and external accounts back to the decedent's estate before calculating the child's share. If the estate isn't sufficient to meet the child's entitlement, in some cases he can legally seek the difference from those who received the gifts.
        \item \textbf{Community property rights} regime means each spouse is entitled to half of the estate earned during the marriage. Gifts and inheritances received before or during the marriage may be held separate. Assets not distributed under community property rights are distributed according to the will.
        \item \textbf{Separate property rights} regime means each spouse owns and controls his or her property. Each spouse may, barring forced heirship rules, bequeath assets as they wish.
    \end{itemize}
\end{flashcard}

\begin{flashcard}[\studyArea]{Balance Sheet Items of Individuals}
    \begin{itemize}
        \item \textbf{Human capital} or \textbf{net employment capital} is the PV of net employment income expected to be generated over the lifetime.
        \item \textbf{Individual liabilities} are the PV of all current and future costs necessary to sustain a given lifestyle. This includes explicit liabilities, such as mortgages or loans, costs of living, and any planned gifts or bequests.
        \item \textbf{Excess capital} is the difference between total assets and total liabilities.
        \item \textbf{Core capital} is the amount of assets necessary to meet all the liabilities plus a reserve for unexpected needs. It must be maintained to meet all present and future liabilities.
    \end{itemize}
\end{flashcard}

\begin{flashcard}[\studyArea]{Relative Value of a Tax-Free Gift}
    \begin{flushleft}
        \begin{align*}
            \text{RV}_{\text{tax-free gift}} &= \frac{\text{FV}_{\text{tax-free gift}}}{\text{FV}_{\text{bequest}}}\\
            &= \frac{(1 + r_{\text{g}} (1 - t_{\text{ig}}))^n}{(1 + r_{\text{e}} (1 - t_{\text{ie}}))^n (1 - T_{\text{e}})}\\
            \\
            \text{where:}\\
            \text{RV}_{\text{tax-free gift}} &= \text{relative after tax-value of a tax-free gift}\\
            r_{\text{g}} &= \text{pre-tax return for the gift receiver}\\
            t_{\text{ig}} &= \text{annual income tax rate for the gift receiver}\\
            r_{\text{e}} &= \text{pre-tax return for the giver}\\
            t_{\text{ie}} &= \text{annual income tax rate for the giver}\\
            T_{\text{e}} &= \text{estate tax rate paid from the estate}
        \end{align*}
    \end{flushleft}
\end{flashcard}

\begin{flashcard}[\studyArea]{Relative Value of a Taxable Gift}
    \vspace{-6mm}
    \begin{flushleft}
        \begin{align*}
            \intertext{When the gift tax is paid by the recipient}
            \text{RV}_{\text{taxable gift}} &= \frac{\text{FV}_{\text{taxable gift}}}{\text{FV}_{\text{bequest}}}\\
            &= \frac{(1 + r_{\text{g}} (1 - t_{\text{ig}}))^n (1 - T_{\text{g}})}{(1 + r_{\text{e}} (1 - t_{\text{ie}}))^n (1 - T_{\text{e}})}\\
            \intertext{When the gift tax is paid by the giver}
            \text{RV}_{\text{taxable gift}} &= \frac{(1 + r_{\text{g}} (1 - t_{\text{ig}}))^n (1 - T_{\text{g}} T_{\text{g}}T_{\text{e}})}{(1 + r_{\text{e}} (1 - t_{\text{ie}}))^n (1 - T_{\text{e}})}\\
            \\
            \text{where:}\\
            \text{RV}_{\text{tax-free gift}} &= \text{relative after tax-value of a tax-free gift}\\
            r_{\text{g}} &= \text{pre-tax return for the gift receiver}\\
            t_{\text{ig}} &= \text{annual income tax rate for the gift receiver}\\
            r_{\text{e}} &= \text{pre-tax return for the giver}\\
            t_{\text{ie}} &= \text{annual income tax rate for the giver}\\
            T_{\text{e}} &= \text{estate tax rate paid from the estate}\\
            T_{\text{g}} &= \text{gift tax rate paid by either the giver or receiver}
        \end{align*}
    \end{flushleft}
\end{flashcard}

\begin{flashcard}[\studyArea]{Generation Skipping}
    \begin{flushleft}
        Without generation-skipping transfer taxes, and in the U.S., transferring assets to a third generation avoids possible double taxation. When the first generation transfers assets to the second, it is subject to taxes. Then the assets are subject to a second taxation when the second transfers to the third.
    \end{flushleft}
\end{flashcard}

\begin{flashcard}[\studyArea]{Valuation Discounts}
    \begin{flushleft}
        Valuation discounts can reduce the value of wealth transfers, and the associated transfer taxes. High net worth individuals will utilize them whenever possible.\newline

        Valuations of private businesses use discount rates and assumptions from comparable public companies. Then, lack of liquidity and minority interest discounts are applied.
    \end{flushleft}
\end{flashcard}

\begin{flashcard}[\studyArea]{Relative Value of a Charitable Gift}
    \begin{flushleft}
        \begin{align*}
            \text{RV}_{\text{charitable gift}} &= \frac{\text{FV}_{\text{charitable gift}}}{\text{FV}_{\text{bequest}}}\\
            &= \frac{(1 + r_{\text{g}})^n + T_{\text{oi}}(1 + r_{\text{e}} (1 - t_{\text{ie}}))^n (1 - T_{\text{e}})}{(1 + r_{\text{e}} (1 - t_{\text{ie}}))^n (1 - T_{\text{e}})}\\
            \\
            \text{where:}\\
            \text{RV}_{\text{tax-free gift}} &= \text{relative after tax-value of a tax-free gift}\\
            r_{\text{g}} &= \text{pre-tax return for the gift receiver}\\
            r_{\text{e}} &= \text{pre-tax return for the giver}\\
            t_{\text{ie}} &= \text{annual income tax rate for the giver}\\
            T_{\text{e}} &= \text{estate tax rate paid from the estate}\\
            T_{\text{oi}} &= \text{tax rate on ordinary income}
        \end{align*}
    \end{flushleft}
\end{flashcard}

\begin{flashcard}[\studyArea]{Trusts}
    \begin{itemize}
        \item Means by which a grantor (settlor) can transfer assets to beneficiaries outside of the probate process. The trustee holds the assets and manages them in the best interests of the beneficiaries.
        \item Different types of trusts are
            \begin{itemize}
                \item \textbf{Revocable trusts} can be rescinded and assets transferred back to the settlor. The settlor is the legal owner for tax and reporting purposes, and creditors can make claims against the assets.
                \item \textbf{Irrevocable trusts} are given to the trustee, who is the owner of the assets for tax purposes and is responsible for paying taxes on trust income. Irrevocable trusts are protected from claims against the settlor.
                \item \textbf{Fixed trusts} have a predetermined pattern of distributions set in the trust documents. The trustee is responsible for distributing assets.
                \item \textbf{Discretionary trusts} have distributions determined by the trustee to the greatest benefit for the beneficiaries. A letter of wishes conveys general direction. Beneficiaries have no legal right to either the income or assets.
                \item \textbf{Spendthrift trusts} transfer assets to a beneficiary who is too young to manager assets. Gives a means for the settlor to transfer assets outside of the probate process while maintaining control over the distributions.
            \end{itemize}
        \item In some countries, trusts legally transfer assets, but not for tax purposes. In this case, the settlor is responsible for paying taxes.
    \end{itemize}
\end{flashcard}

\begin{flashcard}[\studyArea]{Life Insurance as a Wealth Transfer Mechanism}
    \begin{flushleft}
        Most of the time life insurance proceeds pass to beneficiaries without tax consequences, and depending on the jurisdiction the policy might give tax-free accumulation of wealth. This makes life insurance an efficient method of transferring assets.\newline

        Can be used in combination with a trust by making a trust the direct beneficiary of the life policy. This avoids the probate process.
    \end{flushleft}
\end{flashcard}

\begin{flashcard}[\studyArea]{Residence and Source Jurisdictions}
    \begin{itemize}
        \item \textbf{Income taxes}
            \begin{itemize}
                \item Under source jurisdiction, a country levies taxes on all income generated within its borders, whether by citizens or foreigners.
                \item Under residence jurisdiction, a country taxes resident income regardless of where it's generated.
            \end{itemize}
        \item \textbf{Wealth transfer taxes}
            \begin{itemize}
                \item Under source jurisdiction, transfer taxes are levied on assets located within or transferred within a country, whether by citizens or foreigners.
                \item Under residence jurisdiction, citizens and residents pay transfer taxes, regardless of the worldwide location of the assets.
            \end{itemize}
        \item \textbf{Exit taxes}
            Based on the gains on assets leaving a country when renouncing citizenship, as if they were being sold. Could include a tax on income earned for a period---called a shadow period---following the expatriation.
    \end{itemize}
\end{flashcard}

\begin{flashcard}[\studyArea]{Residence-Source Conflict Policies}
    \begin{itemize}
        \item \textbf{Credit method.} The residence country allows the individual to take a tax credit for taxes paid to a source country. The tax rate paid on foreign source income is the greater of the domestic and source tax rates.
            \[
                T_{\text{credit}} = \max(T_{\text{residence}}, T_{\text{source}})
            \]
        \item \textbf{Exemption method.} The residence country charges no income tax on income generated in a foreign country that enforces source jurisdiction.
        \item \textbf{Deduction method.} The individual pays full tax to the source country and can deduct the amount of taxes paid to the source country in calculating total world-wide income.
            \[
                T_{\text{deduction}} = T_{\text{residence}} + T_{\text{source}}(1 - T_{\text{residence}})
            \]
    \end{itemize}
\end{flashcard}

\begin{flashcard}[\studyArea]{Risks Associated with Single-Asset Positions}
    \begin{itemize}
        \item \textbf{Systematic risk} can't be diversified away through holding risky assets. In single-factor CAPM, it's beta. In multifactor models there would be more than one systematic risk, e.g., unexpected changes in the business cycle or inflation.
        \item \textbf{Company-specific risk} is nonsystematic risk that can be diversified away. It comes from specific investments, but not the market. Increases standard deviation without additional expected return.
        \item \textbf{Property-specific risk} for real estate is the direct counterpart to company-specific risk for a company. Might arise, e.g., from environmental pollution or loss of a key tenant.
    \end{itemize}
\end{flashcard}

\begin{flashcard}[\studyArea]{Objectives and Constraints for Single-Asset Positions}
    \begin{itemize}
        \item \textbf{Objectives}
            \begin{itemize}
                \item Reduce the risk caused by the wealth concentration.
                \item Generate liquidity to meet diversification or spending needs.
                \item Optimize tax efficiency to maximize after-tax ending value.
            \end{itemize}
        \item \textbf{Constraints}
            \begin{itemize}
                \item Stock ownership may be expected or required to be maintained for a certain length of time.
                \item A desire for control through majority ownership.
                \item Desire to create wealth, e.g., by an entrepreneur who wishes to take high risk in order to build his business.
                \item The asset may have other uses, e.g., real estate owned personally could be a key asset in another business of the owner.
            \end{itemize}
    \end{itemize}
\end{flashcard}

\begin{flashcard}[\studyArea]{Tax and Liquidity Considerations for Concentrated Positions}
    \begin{flushleft}
        Sale of a concentrated position may trigger a large capital gains tax liability, as they are often accumulated over years with low or zero cost basis. A plan to defer, reduce or eliminate the tax may be in order.\newline

        Illiquidity and high transaction costs may arise if a public company is trading with insufficient volume, or if finding a buyer for a private business or real estate. The intended use may affect the price.
    \end{flushleft}
\end{flashcard}

\begin{flashcard}[\studyArea]{Capital Market and Institutional Constraints on Ability to Reduce a Concentrated Position}
    \begin{itemize}
        \item \textbf{Margin lending rules} limit the percentage of asset's value that can be borrowed. Derivatives can be used to reduce risk of asset position and increase borrowing percentage. Rule-based systems are rigid and specify and exact percentage, while risk-based systems consider the economics.
        \item \textbf{Securities law and regulations} may define owner as an insider and impose restrictions, regulations, and reporting requirements.
        \item \textbf{Contractual restrictions and employer mandates} may impose restrictions, e.g., minimum holding periods or black out dates for sales.
        \item \textbf{Capital market limitations} via market structure and regulation can have indirect consequences. Monetization strategies often need OTC contracts to hedge risk and increase LTV ratio. These contracts may be unavailable.
    \end{itemize}
\end{flashcard}

\begin{flashcard}[\studyArea]{Goal-Based Decision Process}
    \begin{enumerate}
        \item Allocate funds to a personal risk bucket to protect from poverty or a drastic decline in lifestyle. Low-risk assets, like money market, CDs, and personal residence are held here. Safety is emphasized and a below-market return is likely.
        \item Next, a market risk bucket maintains current standard of living. Assets in this bucket are allocated to stocks and bonds expecting a market return.
        \item Remaining funds are allocated to an aspirational risk bucket holding positions such as private business, concentrated stock, real estate, and other risky positions.
    \end{enumerate}
\end{flashcard}

\begin{flashcard}[\studyArea]{Considerations for Wealth Transfer of Concentrated Positions}
    \begin{itemize}
        \item Advisors should work with clients before significant unrealized gains occur.
        \item Donating assets with unrealized gains to charity is tax-free, even if there are gains.
        \item An estate tax freeze transfers future appreciation and tax liability to a future generation. Usually involves a partnership or corporate structure. Gift tax is due when the transfer is made, but future appreciation in value is exempt from future estate and gift taxes in the giver's estate.
    \end{itemize}
\end{flashcard}

\begin{flashcard}[\studyArea]{Three Techniques to Manage Concentrated Positions}
    \begin{itemize}
        \item \textbf{Sell the asset.} This will trigger a tax liability and loss of control.
        \item \textbf{Monetize the asset.} Borrow against its value and use the loan proceeds to accomplish client objectives.
        \item \textbf{Hedge the asset value.} Often done using derivatives to limit downside risk.
    \end{itemize}
\end{flashcard}

\begin{flashcard}[\studyArea]{Monetizing Concentrated Stock Positions}
    \begin{itemize}
        \item \textbf{Short sale against the box.} This creates a ``riskless'' position earning the risk-free rate. The short sale proceeds can be used to meet objectives, or as collateral to borrow nearly 100\% LTV of the hedged position.
        \item \textbf{Equity forward sale contract.} Enter a forward contract to sell the position. There is no upside or downside risk.
        \item \textbf{Forward conversion with options.} A pair of options is used to hedge the position, selling calls and buying puts with the same strike.
        \item \textbf{Total return equity swap.} Pay the total return on the position and receive LIBOR\@. The investor is fully hedged.
    \end{itemize}
\end{flashcard}

\begin{flashcard}[\studyArea]{Prepaid Variable Forwards}
    \begin{flushleft}
        PVFs are similar to a collar and a loan in one transaction. The investor receives a set price per share now in exchange for delivery of shares in the future.\newline

        The contract could specify certain amounts of shares to be delivered based on the resulting price of the shares.
    \end{flushleft}
\end{flashcard}

\begin{flashcard}[\studyArea]{Tax-Optimization Equity Strategies for Concentrated Positions}
    \begin{itemize}
        \item \textbf{Index tracking with active tax management.} Cash from a monetized concentrated stock position is invested to track a broad market index on a pre-tax basis and outperform the index on an after-tax basis.
        \item \textbf{Completeness portfolio.} Structures the other portfolio assets for greatest diversification benefit to complement the concentrated position.
    \end{itemize}
\end{flashcard}

\begin{flashcard}[\studyArea]{Cross Hedge Possibilities for Concentrated Positions}
    \begin{itemize}
        \item Short shares of a highly correlated stock which will increase to offset decreases in the concentrated position and vice versa.
        \item Short an index that is highly correlated. This (and shorting a single stock) introduces company-specific risk. A negative event could affect the concentrated position with no offsetting value on the short position.
        \item Purchase puts on the concentrated position.
    \end{itemize}
\end{flashcard}

\begin{flashcard}[\studyArea]{Exchange Funds}
    \begin{flushleft}
        A number of investors who have concentrated positions pool them together and take pro rata shares of the new fund. They now all participate in a diversified portfolio and have deferred any tax event until shares of the fund are sold.
    \end{flushleft}
\end{flashcard}

\begin{flashcard}[\studyArea]{Considerations for Exit Strategies for Privately Held Businesses}
    \begin{itemize}
        \item Value of the business
        \item Tax rates that would apply to the potential exit strategy.
        \item Availability and terms of credit, as borrowing may be involved in financing any transaction.
        \item Buying power of potential buyers.
        \item Currency values if the transaction involves foreign currency.
    \end{itemize}
\end{flashcard}

\begin{flashcard}[\studyArea]{Possible Exit Strategies for Privately Held Businesses}
    \begin{itemize}[nosep]
        \item \textbf{Strategic buyers} take a buy and hold perspective and generally offer the highest price.
        \item \textbf{Financial buyer} or \textbf{financial sponsor} is often a PE fund planning to restructure, add value, and resell. Generally purchase more mature businesses.
        \item \textbf{Recapitalization} is generally used for less mature companies. In leverage recapitalization the owner may retain 20--40\% and sell 60--80\% of his shares back to the company. Continues to manage with a financial stake.
        \item \textbf{Sale to management or key employees} has drawbacks in a discounted price, lack of financing, possible need for future success, and possible failed negotiations damaging relationship.
        \item \textbf{Divestiture, sale, or disposition of non-core business assets} sells nonessential assets and then directs the company to use proceeds to pay a large dividend or repurchase stock.
        \item \textbf{Sale of gift to family members} could be structured with tax advantages such as estate tax freeze or limited partnership valuation discounts.
        \item \textbf{Personal line of credit secured by company shares} means owner can borrow from company and pledge stock as collateral.
        \item \textbf{Initial public offering} sells a portion of owner shares to the public and transforms remaining shares into liquid public shares. Generally assumed owner will retain significant ownership stake.
        \item \textbf{Employee stock ownership plan} sells stock to the ESOP which in turns sells to employees. In a leveraged ESOP, the company borrows to finance the stock purchase.
    \end{itemize}
\end{flashcard}

\begin{flashcard}[\studyArea]{Strategies for Managing Concentrated Real Estate Positions}
    \begin{itemize}
        \item \textbf{Mortgage financing} can raise funds without loss of control of the property. With a nonrecourse loan, the lender's only recourse is to seize the property. The borrower effectively has a put option on the property. The borrower effectively has a put option on the property.
        \item \textbf{Donor-advised fund} or \textbf{charitable trust} can allow the property owner to take a tax deduction, gift more money to the charity, and influence the use of the donation.
        \item \textbf{Sale and leaseback} can provide immediate funds while retaining use of the property.
    \end{itemize}
\end{flashcard}

\begin{flashcard}[\studyArea]{Human Capital}
    \begin{flushleft}
        Measure of the individual's lifetime earning capacity. Present value of expected future labor income from salary, wages, bonuses, etc. Should be treated as another asset.
        \begin{align*}
            \text{HC}_j &= \sum_{t=j+1}^n \frac{\hat{I}_{t}}{(1 + r)^{t-j}}\\
            \\
            \text{where:}\\
            j &= \text{individual's current age}\\
            \text{HC}_j &= \text{individual's human capital at age $j$}\\
            \hat{I}_t &= \text{expected income in year $t$}\\
            n &= \text{expected remaining years of life (at age $j$)}\\
            r &= \text{discount rate reflecting inherent risk of income}
        \end{align*}
    \end{flushleft}
\end{flashcard}

\begin{flashcard}[\studyArea]{Relationship Between Human Capital and Financial Capital}
    \begin{flushleft}
        \begin{center}
            \begin{tikzpicture}
                \draw[->] (0, 0) -- (0, 7) node[above] {\$};
                \draw[->] (0, 0) -- (10, 0) node[right] {Age};
                \draw[dashed] (8, 0) -- (8, 7);
                \node [right, rotate=90] at (8.25, 2)  {Retirement};
                \node [below] at (4, 0) {Accumulation Phase};
                \draw[->] (2, -0.25) -- (0.25, -0.25);
                \draw[->] (6, -0.25) -- (7.75, -0.25);
                \draw[dashed] (0.25, 0.25) .. controls (3, 1.5) and (5, 2) .. (7.75, 6);
                \draw (0.25, 6) .. controls (3, 4.8) and (5, 3.74) .. (7.75, 0.25);
                \draw[->] (5.75, 5) node[above left] {Financial Capital} -- (6.5, 4.5);
                \draw[->] (5.75, 1) node[below left] {Financial Capital} -- (6.5, 1.5);
            \end{tikzpicture}
        \end{center}
    \end{flushleft}
\end{flashcard}

\begin{flashcard}[\studyArea]{Considerations for Allocating Life Insurance and Human Capital}
    \begin{itemize}
        \item Individual's risk aversion will affect riskiness of the portfolio.
        \item Asset allocation and life insurance holdings both affect investor utility and should be solved for jointly.
        \item Strong correlation between HC and risky assets reduces the allocation to risky assets in the FC as well as need for life insurance. This is caused by high correlation raising risk, increasing the discount rate, lowering HC, and thus lowering life insurance.
        \item An asset allocation is expected to meet both pre- and post-death objectives. The larger the post-death objectives, the greater the need for life insurance.
        \item Those with relatively more FC have less HC to insure and less need for life insurance.
        \item As FC increases over time, the risk take with the FC should be positively related to the riskiness of the HC.
    \end{itemize}
\end{flashcard}

\begin{flashcard}[\studyArea]{Significant Risks of Investors in Retirement}
    \begin{itemize}
        \item Financial risk is due to changes in financial market prices or conditions. In the short term, holding assets witb low volatility, lowers portfolio volatility, but in the long run may produce lower return.
        \item Longevity risk is the risk of outliving the FC assets, and it rises now that HC is zero. This risk has been rising as life expectancies have been increasing. While public and private DB plans have been becoming less significant.
        \item Spending uncertainty risk is the inability to precisely predict how much will be needed during retirement or how long it will be needed.
    \end{itemize}
\end{flashcard}

\begin{flashcard}[\studyArea]{Fixed and Variable Annuities}
    \begin{itemize}
        \item Fixed annuity makes a constant nominal payment for life and leaves the beneficiary exposed to declining real value over time.
        \item A variable annuity is also for life, but the size of the payments is linked to an underlying portfolio, market, or assets. This may give some inflation protection over time.
    \end{itemize}
\end{flashcard}
\end{document}
