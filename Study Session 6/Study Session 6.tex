\documentclass[../custom]{flashcards}
\usepackage{enumitem}
\usepackage{amsmath}
\usepackage{booktabs}
\usepackage{array}
\usepackage{multirow}

\newcommand{\studyArea}{Capital Market Expectations}

\def\labelitemii{$\circ$}
\def\labelitemiii{$\diamond$}
\def\labelitemiv{$\cdot$}

\begin{document}
\cardfrontstyle{headings}
\cardfrontfoot{Study Session 6}

\begin{flashcard}[\studyArea]{Steps to Formulate Capital Market Expectations}
    \begin{enumerate}[label=Step \arabic*.]
        \item Determine expectations needed according to investor's tax status, allowable asset classes, and time horizon. Time horizon is particularly important.
        \item Look at drivers of historical performance to establish a range for future performance. The drivers can be used to forecast future performance and compare it to past results.
        \item Identify the valuation model used and its requirements.
        \item Collect the best data possible as faulty data will lead to faulty conclusions. These issues should considered in data evaluation.
            \begin{itemize}[nosep]
                \item Calculation methodologies and error rates.
                \item Data collection techniques and data definitions.
                \item Investability and correction for free float.
                \item Turnover in index components.
                \item Potential biases.
            \end{itemize}
        \item Use experience and judgment to interpret investment conditions and decide on values for required inputs.
        \item Formulate capital market expectations. Any assumptions and rationales should be required. Determine if what was specified in Step 1 has been provided.
        \item Monitor performance and use it to refine the process. If actual performance varies significantly from forecasts, the process and model should be refined.
    \end{enumerate}
\end{flashcard}

\begin{flashcard}[\studyArea]{Problems with Producing Capital Market Forecasts}
    \begin{itemize}[itemsep=.2\itemsep]
        \item Limitations to using economic data include time lag between collection and distribution, data are often revised, data definitions and methodologies change, and data indices are rebased over time.
        \item Data measurement errors and biases include transcription errors, survivorship bias, and appraisal (smoothed) data instead of actual returns. Appraisal data gives correlations and standard deviations that are biased downwards.
        \item Values from historical data must be adjusted as economic, political, regulatory, and technological environments change. These are regime changes and result in nonstationary data.
        \item Using ex post data to determine ex ante risk and return can be problematic.
        \item Using historical data can lead to patterns that are unlikely to occur in the future. Data mining---in which variables appear to have a relationship by change---is an example of this.
        \item Forecasts may not account for conditioning information. Accounting for current conditions avoids assuming economic relationships will persist over time.
        \item Misinterpretation of correlations versus causality. An alternative to correlation for uncovering predictive relationships is a multiple regression.
        \item Falling into psychological traps including anchoring, status quo, confirming evidence, overconfidence, prudence, and recallability traps.
        \item Model and input uncertainty.
    \end{itemize}
\end{flashcard}

\begin{flashcard}[\studyArea]{Statistical Tools for Capital Market Expectations}
    \begin{itemize}
        \item \textbf{Projecting historical data} is most straightforward. Historical mean return, standard deviation, and correlations are projected into the future. Arithmetic mean is used for standard deviation and for single period returns, but geometric mean is more accurate for multiple periods.
        \item \textbf{Shrinkage estimators} are weighted averages of historical data and other estimates which shrink influence of historical outliers. Mean return and covariance are the parameters adjusted most often. Useful for small datasets with unreliable historical estimates.
        \item \textbf{Time series analysis} forecasts a variable using previous values. Can be used to forecast means or variances. Stocks, futures, and foreign exchange exhibit volatility clustering in which high or low volatility is persistent across periods. That is,
            \[
                \sigma^2_t = \theta \sigma_{t-1}^2 + (1 - \theta) \varepsilon_t^2
            \]
            where $\sigma^2_t$ is current period volatility and $\theta$ is a rate of decay.
        \item \textbf{Multifactor models} are used to forecast returns and covariances and reduce the forecasting procedure to a common set of factors. Eliminates noise present in a sample and ensures consistent forecasts given a consistent covariance matrix.
    \end{itemize}
\end{flashcard}

\begin{flashcard}[\studyArea]{Gordon Growth Model}
    \begin{flushleft}
        \begin{align*}
            \hat{R}_i &= \frac{\text{Div}_0 (1 + g)}{P_0} + g = \frac{\text{Div}_1}{P_0} + g\\
            \text{where:}\\
            \hat{R}_i &= \text{expected return on stock $i$}\\
            \text{Div}_t &= \text{dividend in period $t$}\\
            P_0 &= \text{current stock price}\\
            g &= \text{growth rate}
        \end{align*}

        Advantage is the theoretical accuracy of modeling based on future cash flows. Disadvantage is not accounting for current market conditions. Thus more suitable for long-term valuation.
    \end{flushleft}
\end{flashcard}

\begin{flashcard}[\studyArea]{Grinold-Kroner Cash Flow Model}
    \begin{flushleft}
        \begin{align*}
            \hat{R}_j &= \frac{D_1}{P_0} + i + g - \Delta S + \Delta \frac{P}{E}\\
            \text{where:}\\
            \hat{R}_j &= \text{expected return on stock $j$ (compound annual growth rate)}\\
            \frac{D_1}{P_0} &= \text{dividend yield}\\
            i &= \text{expected inflation}\\
            g &= \text{real growth rate}\\
            \Delta S &= \text{percentage change in shares outstanding}\\
            \Delta \frac{P}{E} &= \text{percentage change in P/E ration (repricing term)}
        \end{align*}
    \end{flushleft}
\end{flashcard}

\begin{flashcard}[\studyArea]{Components of the Grinold-Kroner Model}
    \begin{itemize}
        \item \textbf{Expected income return} is the current yield investors expect to receive.
            \[
                \text{expected income return} = \frac{D_1}{P_0} - \Delta S
            \]
        \item \textbf{Expected nominal earnings growth} is real growth in stock price plus expected inflation.
            \[
                \text{expected nominal earnings growth} = i + g
            \]
        \item \textbf{Repricing return} is expected change in the P/E ratio.
            \[
                \text{expected repricing return} = \Delta \frac{P}{E}
            \]
        \item Grinold-Kroner model is the sum of these components.
            \begin{align*}
                \hat{R}_i &= E(\text{income return}) + E(\text{nominal earnings growth}) + E(\text{repricing return})\\
                &= \left ( \frac{D_1}{P_0} - \Delta S \right ) + (i + g) + \left ( \Delta \frac{P}{E} \right )
            \end{align*}
    \end{itemize}
\end{flashcard}

\begin{flashcard}[\studyArea]{Risk Premium Approach}
    \begin{itemize}
        \item Estimates a bond yield, $\hat{R}_B$, and adds an equity risk premium.
        \item To find $\hat{R}_B$, risk premia are added together.
            \begin{align*}
                \hat{R}_B &= \text{real risk-free rate} + \text{inflation risk premium} + \text{default risk premium}\\
                &+ \text{liquidity risk premium} + \text{maturity risk premium} + \text{tax premium}
            \end{align*}
        \item Inflation premium represents loss in purchasing power over time. Can be measured by comparing inflation-indexed bonds to non-inflation-indexed bonds.
        \item Default risk premium compensates for non-payment. Can be measured by comparing yields on bonds with different credit risks.
        \item Liquidity risk premium compensates for holding illiquid bonds.
        \item Maturity risk premium reflects yield differences in different maturities.
        \item Tax premium accounts for different tax treatments.
    \end{itemize}
\end{flashcard}

\begin{flashcard}[\studyArea]{International Capital Asset Pricing Model}
    \begin{flushleft}
        \begin{align*}
            \hat{R}_i &= R_F + \beta_i (\hat{R}_M - R_F)\\
            \text{where:}\\
            \hat{R}_i &= \text{expected return on asset $i$}\\
            R_F &= \text{risk-free return}\\
            \beta_i &= \text{sensitivity of asset $i$ returns to global market}\\
            \hat{R}_M &= \text{expected return on the global investable market}
        \end{align*}

        Using $\beta_i = \rho_{i, M} \sigma_i / \sigma_M$, we can solve for the risk premium as
        \[
            \text{RP}_i = \rho_{i, M} \sigma_i \frac{\hat{R}_M - R_F}{\sigma_M}
        \]
    \end{flushleft}
\end{flashcard}

\begin{flashcard}[\studyArea]{Singer and Terhaar Analysis}
    \begin{flushleft}
        Adjustment to ICAPM for illiquidity and market segmentation. Liquidity is not typically a concern for developed markets, but can be for real estate and private equity. The latter may have lock-up periods.\newline

        In segmented markets, capital doesn't flow freely across borders like in integrated markets. Frequently caused by government investment restrictions. Two assets with the same risk can have different returns because capital cannot flow to the higher return asset.\newline

        To adjust for partial market segmentation, find the ERP for a global market and a local market and take a weighted average. The weighting is the degree of integration.
    \end{flushleft}
\end{flashcard}

\begin{flashcard}[\studyArea]{Survey and Panel Methods}
    \begin{flushleft}
        In the survey method, economists and analysts are polled about their expectations for the economy or capital markets.\newline

        If the group polled is fairly constant over time, this is the panel method.
    \end{flushleft}
\end{flashcard}

\begin{flashcard}[\studyArea]{Cyclical and Trend Economic Growth}
    \begin{itemize}
        \item Cyclical economic growth
            \begin{itemize}
                \item Inventory cycle is 2--4 years.
                    \begin{itemize}
                        \item Measured using inventory-to-sales ratio
                        \item Increases as businesses gain confidence in the economy and add to inventories.
                    \end{itemize}
                \item Business cycle is 9--11 years with five phases.
                    \begin{enumerate}
                        \item Initial recovery
                        \item Early upswing
                        \item Late upswing
                        \item Slowdown
                        \item Recession
                    \end{enumerate}
            \end{itemize}
        \item Trend-growth
            \begin{itemize}
                \item GDP is usually measured in real terms.
                \item Output gap is difference between a long-term trending GDP and current GDP.
                \item A recession is decreases in GDP over two consecutive quarters.
            \end{itemize}
    \end{itemize}
\end{flashcard}

\begin{flashcard}[\studyArea]{Characteristics of Business Cycle Phases}
    \vspace{-3mm}
    \begin{itemize}[nosep]
        \item \textbf{Initial recovery}
            \begin{itemize}[nosep]
                \item Duration of a few months
                \item Falling inflation and large output gap
                \item Low or falling short-term interest rates
                \item Bond yields bottoming out and rising stock prices
            \end{itemize}
        \item \textbf{Early upswing}
            \begin{itemize}[nosep]
                \item Duration of a year to several years
                \item Increasing growth with low inflation and output gap is narrowing
                \item Rising short-term interest rates
                \item Flat or rising bond yields and rising stock prices
            \end{itemize}
        \item \textbf{Late upswing}
            \begin{itemize}[nosep]
                \item Inflation increases and output gap eliminated
                \item Central bank limits growth of money supply
                \item Rising bond yields and stock prices, but increased risk and volatility
            \end{itemize}
        \item \textbf{Slowdown}
            \begin{itemize}[nosep]
                \item Duration of a few months to longer than a year
                \item Inflation still strong and falling inventory levels
                \item Bond yields have peaked, may be falling and yield curve may invert
                \item Falling stock prices
            \end{itemize}
        \item \textbf{Recession}
            \begin{itemize}[nosep]
                \item Duration of six months to a year
                \item Inflation tops out and large declines in inventory
                \item Falling short-term interest rates
                \item Falling bond yields and stock prices increase in later stages
            \end{itemize}
    \end{itemize}
\end{flashcard}

\begin{flashcard}[\studyArea]{Effects of Inflation on Asset Classes}
    \begin{flushleft}
        Inflation is negative for bonds and bonds will decline in price during a expansion and rise in price during an recession.\newline

        Low inflation can be positive for equities. Equities provide an inflation hedge when inflation is moderate and when prices can be passed on to the consumer. Inflation above 3\% can be bad because the central bank may restrict economic growth.\newline

        Deflation results in declining economic growth and asset prices. Also reduces value of real assets financed with debt. If leverage is used, declines in value lead to steeper declines in the equity position.
    \end{flushleft}
\end{flashcard}

\begin{flashcard}[\studyArea]{Consumer and Business Spending in the Business Cycle}
    \begin{flushleft}
        Consumer confidence increases during recovery and consumers begin to spend more. At the same time, stock prices rise. Consumers continue spending until the economy shows definite signs that it has peaked and reversed. Then they save more until the cycle starts over.\newline

        Business spending is more volatile, particularly on inventory and investments. The peak of inventory spending signals a decline.
    \end{flushleft}
\end{flashcard}

\begin{flashcard}[\studyArea]{Taylor Rule}
    \begin{flushleft}
        \begin{align*}
            r_{\text{target}} &= r_{\text{neutral}} + \frac{(\text{GDP}_{\text{expected}} - \text{GDP}_{\text{trend}}) + (i_{\text{expected}} - i_{\text{target}})}{2}\\
            \text{where:}\\
            r_{\text{target}} &= \text{short-term interest rate target}\\
            r_{\text{neutral}} &= \text{neutral short-term interest rate}\\
            \text{GDP}_{\text{expected}} &= \text{expected GDP growth rate}\\
            \text{GDP}_{\text{trend}} &= \text{long-term GDP growth rate trend}\\
            i_{\text{expected}} &= \text{expected inflation rate}\\
            i_{\text{target}} &= \text{target inflation rate}
        \end{align*}
    \end{flushleft}
\end{flashcard}

\begin{flashcard}[\studyArea]{Effects of Monetary and Fiscal Policy on the Yield Curve}
    \begin{tabular}
        {>{\raggedright}m{0.8in}
         >{\raggedright}m{0.8in}
         >{\raggedright}m{1.3in}
         >{\raggedright\arraybackslash}m{1.3in}} \toprule

        & & \multicolumn{2}{c}{\textit{Monetary Policy}} \\

        & & Stimulative & Restrictive \\ \cmidrule(r){3-4}

        \multirow{2}{*}{\textit{Fiscal Policy}} &
        Stimulative &
        Yield curve is steep and economy is likely to grow &
        Yield curve is flat and economy is unclear \\ \addlinespace

        &
        Restrictive &
        Yield curve is moderately steep and economy is unclear &
        Yield curve is inverted and economy is likely to contract \\ \bottomrule
    \end{tabular}
\end{flashcard}

\begin{flashcard}[\studyArea]{Components of Economic Trend Growth Rate}
    \begin{itemize}
        \item Changes in employment levels
            \begin{itemize}
                \item Population growth
                \item Rate of labor force participation
            \end{itemize}
        \item Changes in productivity
            \begin{itemize}
                \item Spending on new capital inputs
                \item Total factor productivity growth
            \end{itemize}
    \end{itemize}
\end{flashcard}

\begin{flashcard}[\studyArea]{Government Policies to Enhance Growth}
    \begin{itemize}
        \item Provide infrastructure needed for growth, but interfere with the economy as little as possible.
        \item Have a responsible fiscal policy. Consistently high budget deficits lead to high inflation. They are also accompanied by trade deficits, which may lead to devaluation of the currency. Also, public borrowing may crowd out private borrowing.
        \item Have tax policies that are transparent, consistently applied, pulled from a wide base, and not overly burdensome.
        \item Promote competition in the marketplace to increase economic efficiency. Technological advances and openness to foreign competition to reduction of tariffs are important factors for growth.
    \end{itemize}
\end{flashcard}

\begin{flashcard}[\studyArea]{Exogenous Shocks}
    \begin{flushleft}
        Exogenous shocks are unanticipated events that occur outside the normal course of an economy. These events are not built into market prices like endogenous events.\newline

        Can be caused by natural disasters, political events, or changes in government policy.\newline

        Negative shocks are more common. Often spread to other countries in a process called contagion.
    \end{flushleft}
\end{flashcard}

\begin{flashcard}[\studyArea]{Macroeconomic, Exchange Rate, and Interest Rate Links}
    \begin{flushleft}
        Macroeconomic links are similarities in business cycles across countries. International trade and capital flows are linked, so a recession in one country dampens exports and investment in another.\newline

        Exchange rate links are formed by adoption or pegging of one country's currency by another. Countries are not always successful in maintaining a peg, and so there is a risk premium with the weaker country having higher interest rates.\newline

        Interest rate differentials reflect differences in economic growth and monetary and fiscal policy. In theory, real interest rate differentials would not exist. Countries with high real interest rates should see their currency increase.
    \end{flushleft}
\end{flashcard}

\begin{flashcard}[\studyArea]{Consideration for Investing in Emerging Markets}
    \begin{itemize}[itemsep=.2\itemsep]
        \item \textbf{Responsible fiscal and monetary policies.} Deficit to GDP ratio of over 4\% indicates substantial credit risk. Most emerging countries borrow short-term and must refinance frequently.
        \item \textbf{Expected growth.} Expect a growth rate of at least 4\%. Less than that may indicate economy is growing slower than population. Tariffs, tax policies and regulation of foreign investment are all important factors for growth.
        \item \textbf{Reasonable currency values and current account deficits.} Volatile currencies discourage foreign investment, and overvalued currencies can cause excessive borrowing by the government. Current account deficits (imports vs.\ exports) greater than 4\% of GDP mean the deficit must be financed through external borrowing.
        \item \textbf{Not too highly levered.} Too much debt will lead to financial crisis followed by currency devaluations and declines in emerging market asset values. Foreign debt levels greater than 50\% of GDP indicate overleverage. Debt levels greater than 200\% of the current account receipts are also high risk.
        \item \textbf{Foreign exchange reserves relative to short-term debt.} Foreign exchange is necessary to pay back loans. Foreign exchange reserves should be greater than the foreign debt due within one year.
        \item \textbf{Government's stance on structural reform.} A supportive stance is good for investment. Commitment to responsible fiscal policy, competition, and privatization of state-owned businesses encourage growth.
    \end{itemize}
\end{flashcard}

\begin{flashcard}[\studyArea]{Econometric Analysis}
    \begin{flushleft}
        Uses economic theory to formulate the forecasting model. Models can be simple or complex, involving several data items of various time period lags.

        \begin{itemize}
            \item Advantages
                \begin{itemize}
                    \item Once established, can be reused.
                    \item Can be quite complex and may accurately model real world conditions.
                    \item Can provide precise quantitative forecasts of economic conditions.
                \end{itemize}
            \item Disadvantages
                \begin{itemize}
                    \item May be difficult and time intensive to create.
                    \item Proposed model may not be applicable in future time periods.
                    \item Better at forecasting expansions than recessions.
                    \item Requires scrutiny of output to verify validity.
                \end{itemize}
        \end{itemize}
    \end{flushleft}
\end{flashcard}

\begin{flashcard}[\studyArea]{Economic Indicators}
    \begin{flushleft}
        Available from governments, international organizations, and private organizations. Attempt to characterize an economy's phase in the business cycle and are separated into lagging, coincident, and leading indicators. Leading indicators are preferred and can be used individually or as a composite.
        \begin{itemize}
            \item Advantages
                \begin{itemize}
                    \item Available from outside parties.
                    \item Easy to understand and interpret.
                    \item Can be adapted for specific purposes.
                    \item Effectiveness has been verified by academic research.
                \end{itemize}
            \item Disadvantages
                \begin{itemize}
                    \item Not consistently accurate as economic relationships change.
                    \item Can give false signals.
                \end{itemize}
        \end{itemize}
    \end{flushleft}
\end{flashcard}

\begin{flashcard}[\studyArea]{Checklist Approach}
    \begin{flushleft}
        List of questions are used to indicate future growth of the economy. From the answers, subjective judgment is used to formulate a forecast or derive a more formal model.

        \begin{itemize}
            \item Advantages
                \begin{itemize}
                    \item Simple.
                    \item Allows changes in the model over time.
                \end{itemize}
            \item Disadvantages
                \begin{itemize}
                    \item Requires subjective judgment.
                    \item May be time intensive to create.
                    \item May not be able to model complex relationships.
                \end{itemize}
        \end{itemize}
    \end{flushleft}
\end{flashcard}

\begin{flashcard}[\studyArea]{Considerations for Cash Instruments}
    \begin{flushleft}
        Cash managers adjust the maturity and creditworthiness of cash investments depending on their forecasts for interest rates and the economy. The yield curve reflects anticipation of yields over future periods. To earn excess return, a manager must be able to forecast rates better than others. This in part requires anticipation of what the central bank will do.
    \end{flushleft}
\end{flashcard}

\begin{flashcard}[\studyArea]{Considerations for Credit Risk-Free and Credit Risky Bonds}
    \begin{itemize}
        \item Credit risk-free bonds
            \begin{itemize}
                \item Most common credit risk-free bonds are issued by governments in developed countries.
                \item Yield is composed of a real yield and expected inflation.
                \item If the time horizon is short, cyclical and short-term interest rate changes are important.
                \item Higher expected economic growth results in higher yields.
                \item Changes in short-term rates have less predictable effects.
            \end{itemize}
        \item Credit risky bonds
            \begin{itemize}
                \item Most common type are corporate bonds.
                \item Credit risk premium can be estimated as difference between Treasuries and corporates of the same maturity.
                \item In a recession the credit spread widens as default becomes more likely.
            \end{itemize}
    \end{itemize}
\end{flashcard}

\begin{flashcard}[\studyArea]{Considerations for Emerging Market Bonds and Stocks}
    \begin{itemize}
        \item Emerging market government bonds
            \begin{itemize}
                \item Denominated in a non-domestic currency.
                \item Usually denominated in a hard currency which must be obtained to pay P\&I.
                \item Default risk is higher.
                \item Use country risk analysis to asses risk.
            \end{itemize}
        \item Emerging market stocks
            \begin{itemize}
                \item Returns are higher and more variable than developed stocks.
                \item Returns positively correlated with developed business cycle due to trade and capital flows.
                \item Should have an understanding of country and sector patterns.
            \end{itemize}
    \end{itemize}
\end{flashcard}

\begin{flashcard}[\studyArea]{Considerations for Inflation-Indexed Bonds}
    \begin{itemize}
        \item Prices and yields vary with economic conditions and supply and demand.
        \item Yield has been correlated with three economic factors.
            \begin{itemize}
                \item Rises as the real economy expands and falls as it contracts. Primarily because yield tracks short-term interest rates.
                \item Falls as inflation accelerates and more people buy them.
                \item Changes with supply and demand. Small markets mean this has a larger effect.
            \end{itemize}
    \end{itemize}
\end{flashcard}

\begin{flashcard}[\studyArea]{Considerations for Common Stock}
    \begin{itemize}
        \item Aggregate earnings depend on economic growth, which depends on labor force growth, new capital inputs, and TFP growth.
        \item When the government promotes competition, the economy is more efficient which leads to higher long-term growth.
        \item Short-term growth is affected by the business cycle. 
        \item Noncyclical defensive stocks are less affected and have lower risk premia and higher valuations.
        \item P/E ratios are higher in an early expansion with low rates and high earnings prospects.
        \item For cyclical stocks, P/E ratios may be high in a recession if the economy is expected to recover.
    \end{itemize}
\end{flashcard}

\begin{flashcard}[\studyArea]{Considerations for Real Estate}
    \begin{itemize}
        \item Affected by interest rates, inflation, shape of the yield curve, and consumption.
        \item Rates affect supply and demand of properties through mortgage rates.
        \item Rates also determine the discount rate used to value cash flows.
    \end{itemize}
\end{flashcard}

\begin{flashcard}[\studyArea]{Methods to Forecast Exchange Rates}
    \begin{itemize}
        \item \textbf{Purchasing power parity} says a country with higher inflation will have its currency fall in value. Does not hold over the short or medium term, but approximately in the long term.
        \item \textbf{Relative economic strength approach} says a good investment climate or high short-term rates will increase currency value. May be better for forecasting short-run changes in currency value.
        \item \textbf{Capital flows approach} focuses on long-term capital flows, such as those into equity or foreign direct investments.
        \item \textbf{Savings-invstment imbalances approach} assumes an economy must fund investment through savings. If investment is greater than domestic savings, capital must flow in from abroad. In order to keep the capital necessary for a savings deficit, the domestic currency must appreciate.
    \end{itemize}
\end{flashcard}
\end{document}
