\documentclass[../custom,grid]{flashcards}
\usepackage{enumitem}
%\usepackage{booktabs}
%\usepackage{array}
%\usepackage{multirow}
%\usepackage{amsmath}
%\usepackage{tikz}
%\usepackage{float}

\newcommand{\studyArea}{Capital Market Expectations}

\def\labelitemii{$\circ$}
\def\labelitemiii{$\diamond$}
\def\labelitemiv{$\cdot$}

\begin{document}
\cardfrontstyle{headings}
\cardfrontfoot{Study Session 6}

\begin{flashcard}[\studyArea]{Steps to Formulate Capital Market Expectations}
    \begin{enumerate}[label=Step \arabic*.]
        \item Determine expectations needed according to investor's tax status, allowable asset classes, and time horizon. Time horizon is particularly important.
        \item Look at drivers of historical performance to establish a range for future performance. The drivers can be used to forecast future performance and compare it to past results.
        \item Identify the valuation model used and its requirements.
        \item Collect the best data possible as faulty data will lead to faulty conclusions. These issues should considered in data evaluation.
            \begin{itemize}[nosep]
                \item Calculation methodologies and error rates.
                \item Data collection techniques and data definitions.
                \item Investability and correction for free float.
                \item Turnover in index components.
                \item Potential biases.
            \end{itemize}
        \item Use experience and judgment to interpret investment conditions and decide on values for required inputs.
        \item Formulate capital market expectations. Any assumptions and rationales should be required. Determine if what was specified in Step 1 has been provided.
        \item Monitor performance and use it to refine the process. If actual performance varies significantly from forecasts, the process and model should be refined.
    \end{enumerate}
\end{flashcard}

\begin{flashcard}[\studyArea]{Problems with Producing Capital Market Forecasts}
    \begin{itemize}[itemsep=.2\itemsep]
        \item Limitations to using economic data include time lag between collection and distribution, data are often revised, data definitions and methodologies change, and data indices are rebased over time.
        \item Data measurement errors and biases include transcription errors, survivorship bias, and appraisal (smoothed) data instead of actual returns. Appraisal data gives correlations and standard deviations that are biased downwards.
        \item Values from historical data must be adjusted as economic, political, regulatory, and technological environments change. These are regime changes and result in nonstationary data.
        \item Using ex post data to determine ex ante risk and return can be problematic.
        \item Using historical data can lead to patterns that are unlikely to occur in the future. Data mining---in which variables appear to have a relationship by change---is an example of this.
        \item Forecasts may not account for conditioning information. Accounting for current conditions avoids assuming economic relationships will persist over time.
        \item Misinterpretation of correlations versus causality. An alternative to correlation for uncovering predictive relationships is a multiple regression.
        \item Falling into psychological traps including anchoring, status quo, confirming evidence, overconfidence, prudence, and recallability traps.
        \item Model and input uncertainty.
    \end{itemize}
\end{flashcard}
\end{document}
