\documentclass[../custom]{flashcards}
\usepackage{booktabs}
\usepackage{array}
\usepackage{amsmath}
\usepackage{enumitem}

\newcommand{\studyArea}{Alternative Investments}

\def\labelitemii{$\circ$}
\def\labelitemiii{$\diamond$}
\def\labelitemiv{$\cdot$}

\begin{document}
\cardfrontstyle{headings}
\cardfrontfoot{Study Session 13}

\begin{flashcard}[\studyArea]{Roles of Alternative Investments in Portfolio Management}
    \begin{itemize}
        \item Real estate and long-only commodities give exposure to risk and return that stocks and bonds don't.
        \item Hedge funds and managed futures give exposure to special strategies and are dependent on manager skill.
        \item Private equity and distressed securities are a combination of 1 and 2.
    \end{itemize}
\end{flashcard}

\begin{flashcard}[\studyArea]{Common Features of Alternative Investments}
    \begin{itemize}
        \item \textbf{Low liquidity.} Requires attention to determine if they suitable for a given investor. They should have a liquidity premium and higher return.
        \item \textbf{Diversification.} Generally low correlation with stocks and bonds.
        \item \textbf{Due diligence costs.} Costs of researching and monitoring investments can be high. Specialized expertise and skills are often required. Markets often lack transparency, making information difficult to find.
        \item \textbf{Difficult performance evaluation.} Lack of transparency and unique features of strategies make valuation benchmarks difficult to identify.
    \end{itemize}
\end{flashcard}

\begin{flashcard}[\studyArea]{Steps for Due Diligence in Alternative Investment Manager Selection}
    \begin{enumerate}
        \item \textit{Asses the market opportunity offered.} Confirm that there are market inefficiencies for the type of investment in which the manager specializes. Past returns don't justify selecting a manager if there are no opportunities for him to exploit.
        \item \textit{Asses the investment process.} Identify the manager's competitive edge over others. How does his process identify investment opportunities.
        \item \textit{Assess the organization.} Is it stable and well-run? What is the staff turnover?
        \item \textit{Assess the people.} Meet and assess their character, including integrity and competence.
        \item \textit{Assess the terms and structure of the investment.} Consider fee structure, lock-out period, exit strategy and how it aligns the interests of the manager with the investors.
        \item \textit{Asses the service providers.} Investigate outside firms supporting the manager's business (e.g. lawyers or brokers).
        \item \textit{Review documents.} Review the prospectus, audits of manager's reports and other documents. Use legal advice when needed.
        \item \textit{Write-up.} Document the above review process.
    \end{enumerate}
\end{flashcard}

\begin{flashcard}[\studyArea]{Issues to Consider for Individual Investors}
    \begin{itemize}
        \item \textbf{Taxes.} Many alternative investments are structured as limited partnerships which require specialized tax expertise.
        \item \textbf{Suitability.} Time horizon and liquidity needs must be considered as many alternative investments must stay invested for a minimum time period.
        \item \textbf{Commuunication.} Make sure the client understands the implications of a lock-out period or a complex strategy as much as is needed.
        \item \textbf{Decision risk.} Preparedness for losses and the ability to remain in a position at a maximum loss point.
        \item \textbf{Concentrated positions.} Wealthy individuals often have large positions in closely held companies or private residences. These should be considered as preexisting allocations before adding more private equity or real estate investments. These positions may also have unrealized taxable gains that can complicate rebalancing.
    \end{itemize}
\end{flashcard}

\begin{flashcard}[\studyArea]{Characteristics of Real Estate Investment}
    \begin{flushleft}
        Can be classified as direct or indirect. Direct investment includes ownership and management of residences, commercial real estate and agricultural land. Indirect investment generally involves a separate entity to manage the properties. Indirect investments include
        \begin{itemize}
            \item Companies that develop and manage real estate.
            \item Real estate investment trusts, REITs, which are publicly traded shares in a real estate portfolio. Divided into equity and mortgage REITs. Can be purchased in small sizes and are liquid.
            \item Commingled real estate funds, CREFs, which are pooled investments in real estate that are professionally managed and privately held. Are more flexible than REITs, can be either open- or closed-ended. Restricted to wealthy investors and institutions.
            \item Infrastructure funds purchase infrastructure assets (e.g., airports or toll roads). Because they provide a public service and are regulated, tend to have stable, long-term returns. Low correlation with equities. Long-term nature means they are used by institutions with long-term liabilities. Relatively low returns.
        \end{itemize}
        Advantages of real estate investment are low correlation with stocks and bonds, low volatility of returns and an inflation hedge. May offer tax advantages.\newline

        Disadvantages are high information and transaction costs, political risks from tax law changes, high operating expenses, and inability to subdivide investments.
    \end{flushleft}
\end{flashcard}

\begin{flashcard}[\studyArea]{Characteristics of Private Equity Investment}
    \begin{flushleft}
        Investment in non-publicly-traded company. Limited to wealthy individuals or institutions. Often done through pooling funds with others in a private equity fund. Two most important categories are venture capital and buyout funds.\newline

        Two segments of buyout funds are middle-market buyout funds and mega-cap buyout funds. Middle-market funds concentrate on divisions of large publicly-traded companies and small private companies that cannot obtain capital. Mega-cap funds concentrate on taking public firms private.\newline

        Buyout funds add value through some combination of
        \begin{enumerate}
            \item Restructuring company operations and management.
            \item Buying companies for less than intrinsic value.
            \item Creating value by added leverage or restructuring existing debt.
        \end{enumerate}
        Exit strategies are selling the company through private placement or IPOs or through dividend recapitalizations. In the latter, the debt-equity structure is changed by issuing lots of debt and then the company pays a large special dividend to the buyout fund. The fund retains ownership but recoups its cash.\newline

        Private equity is diverse and typically very risky. Venture capitalists are expected to bring both capital and business expertise to a company that might lack management skills.
    \end{flushleft}
\end{flashcard}

\begin{flashcard}[\studyArea]{Characteristics of Investment in Commodities}
    \begin{flushleft}
        Investment can be direct or indirect.
        \begin{itemize}
            \item Direct investment includes direct purchase of the physical commodity (e.g., agricultural products, crude oil, or metals) or the purchase of derivatives on those assets.
            \item Indirect investment can be investment in companies whose business is tied to a commodity (e.g., investing in metal via shares in a mining company).
            \item Direct investment through derivatives is more common as indirect investment doesn't track price changes and buying outright creates storage costs.
        \end{itemize}
        Commodity futures and publicly traded commodity companies are liquid when compared with other alternative investments. Investments in commodities have low correlation with stocks and bonds, business-cycle sensitivity and generally have positive correlation with inflation.
    \end{flushleft}
\end{flashcard}

\begin{flashcard}[\studyArea]{Characteristics of Hedge Fund}
    \begin{flushleft}
        Generally structured to avoid regulation, which allows them to charge large incentive fees. Designed to exploit a perceived market opportunity, often taking long and short positions on a leveraged basis. Many self-describe as exploiting arbitrage opportunities, although this is used loosely to describe low-risk and not risk-free opportunities.
    \end{flushleft}
\end{flashcard}

\begin{flashcard}[\studyArea]{Characteristics of Managed Futures Investments}
    \begin{flushleft}
        Similar to hedge funds in that they are skilled-based and have a similar fee structure. Tend to trade only in derivatives and take positions based on indices. Like hedge funds with a macro focus instead of micro.\newline

        Investment can be done through private commodity pools, managed futures programs as separately managed accounts (called CTA managed accounts), and publicly-traded commodity futures funds available to small investors. Liquidity is higher for publicly-traded funds. Trading strategies used include
        \begin{itemize}
            \item Systematic trading strategies follow rules. Trend following rules may focus on short-, medium-, or long-term trends and are common. Contrarian strategies are less common and have higher diversification.
            \item Discretionary trading strategies use manager judgment and use economic or other criteria.
            \item May invest in all financial markets, only currency markets or a diversified mix of derivatives and underlying commodities.
        \end{itemize}
        The risk characteristics very as with hedge funds. Standard deviation is generally less than half of equities, but greater than bonds. The correlation with equities is low and often negative, higher with bonds but less than 0.5.
    \end{flushleft}
\end{flashcard}

\begin{flashcard}[\studyArea]{Characteristics of Investment in Distressed Securities}
    \begin{flushleft}
        Securities of companies that are in or near bankruptcy. Risk and return depend on skill-based strategy. Can construct subgroups based on structure.
        \begin{itemize}
            \item If structured like a hedge fund, there is more liquidity.
            \item If structured like private equity, there is less liquidity because of fixed terms and closed-ended funds.
            \item The private equity structure is useful when the underlying assets are too illiquid to determine a NAV.
        \end{itemize}
    \end{flushleft}
\end{flashcard}

\begin{flashcard}[\studyArea]{Comparison of Alternative Investment Characteristics}
    \begin{tabular}{
        >{\raggedright}p{.8in}
        >{\raggedright}p{1.2in}
        >{\raggedright}p{1.2in}
        >{\raggedright\arraybackslash}p{1in}}
        \toprule
        & \textit{Types of Investments} & \textit{Risk/Return Features} & \textit{Liquidity}\\ \midrule
        Real estate &
        Residences, commercial real estate, raw land. &
        Large risk component, good diversification. &
        Low.\\ \midrule
        Private equity &
        Preferred shares, VC, buyout funds. &
        Startup and MM have more risk. &
        Low.\\ \midrule
        Buyout funds &
        Large private firms and spin-offs. &
        Less risk than VC, high diversification. &
        Low. \\ \midrule
        Infrastructure funds &
        Public infrastructure assets. &
        Low risk and return, diversification. &
        Low. \\ \midrule
        Commodities &
        Agriculture, oil, metals. &
        Low correlation with stocks, high with inflation. &
        Fairly liquid.\\ \midrule
        Managed futures &
        Generally only derivatives. &
        Risk between stocks and bonds. &
        Lower for private funds.\\ \midrule
        Distressed securities &
        Can be debt or equity. &
        Depends on skill. High returns. &
        Liquid as hedge funds.\\ \bottomrule
    \end{tabular}
\end{flashcard}

\begin{flashcard}[\studyArea]{Comparison of Alternative Investment Benchmarks}
    \begin{tabular}{
        >{\raggedright}p{.8in}
        >{\raggedright}p{1in}
        >{\raggedright}p{1.2in}
        >{\raggedright\arraybackslash}p{1.2in}}
        \toprule
        & \textit{Benchmarks} & \textit{Construction} & \textit{Biases}\\ \midrule
        Real estate &
        NCREIF, NAREIT. &
        NACREIF value weighted, NAREIT cap weighted. &
        Volatility is downward biased.\\ \midrule
        Private equity &
        Cambridge Associates and Thomson Venture Economics. &
        Constructed for buyout and VC. Depends on events. Custom benchmarks. &
        Infrequent repricing means dated values.\\ \midrule
        Commodities &
        Dow Jones-UBS and S\&P Commodity Indices. &
        Assume futures-based strategy. &
        Indices vary widely depending on purpose, makeup, and weight.\\ \midrule
        Managed futures &
        MLMI, CTA indices. &
        MLMI uses trend-following, CTA uses dollar weights. &
        Requires special weighting scheme.\\ \midrule
        Distressed securities &
        Similar to long-only hedge fund benchmarks. &
        Either equal or based on AUM. Selection criteria varies. &
        Self-reporting, backfill, inclusion, popularity, survivorship biases.\\ \bottomrule
    \end{tabular}
\end{flashcard}

\begin{flashcard}[\studyArea]{Hedge Fund Benchmarks}
    \begin{flushleft}
        Many differences in indices as style classifications vary. Index providers compose indices using the following.
        \begin{itemize}
            \item \textit{Selection criteria} can include assets under management, track record and new investment restrictions.
            \item \textit{Style classification} varies as to how how a fund is classified by style.
            \item \textit{Weighting schemes} are usually equally weighted or based on assets under management.
            \item \textit{Rebalancing rules} must be defined for equally weighted indices and the frequency can be from monthly to annually.
            \item \textit{Investabilty} depends on reporting frequency. Some indices are not investable, but independent firms make modifications to provide investable proxies.
        \end{itemize}
        Some indices list the funds comprising the index, and some do not. Some report monthly and some report daily.
        \begin{itemize}
            \item Providers of daily indices include Hedge Fund Research (HFR), Dow Jones and S\&P. The DJ and S\&P explicitly list funds and use equal weighting.
            \item Monthly indices include CISDM, equally weighted indices covering hedge funds and managed futures, EACM100 Index, an equally weighted index of 100 funds, and various benchmarks for different strategies based on assets under management from Credit Suisse.
        \end{itemize}
    \end{flushleft}
\end{flashcard}

\begin{flashcard}[\studyArea]{Hedge Fund Benchmark Biases}
    \begin{itemize}
        \item \textbf{Relevance of past data} may be questionable. Hedge funds are based on manager skill, and past returns may be a reflection of that skill which may not be present today. Funds with a particular style do have similar returns, and individual managers don't beat their style group.
        \item \textbf{Popularity bias} can occur when one fund attracts a lot of capital in a value weighted index. This overweights its presence in the index, without being caused by return on investment.
        \item \textbf{Survivorship bias} occurs when indices drop funds with poor performance, causing an upward bias in reported values. It can be as high as 1.5--3\% per year. The bias is lower for event-drive strategies and higher for hedged equity strategies.
        \item \textbf{Stale price bias} varies depending on the markets used. When a fund uses markets with infrequent trading, pricing and appraisal can lag causing understatement of volatility. Evidences suggests this is not a large problem.
        \item \textbf{Backfill or inclusion bias} arises from filling in past data. Managers will often only supply data if it benefits them, causing an upward bias in past returns.
    \end{itemize}
\end{flashcard}

\begin{flashcard}[\studyArea]{Effects of Adding Real Estate to an Existing Portfolio}
    \begin{itemize}
        \item  High risk-adjusted performance is possible because of low liquidity, large lot sizes, immobility, high transaction costs, and low information transparency.
        \item Adding direct real estate or REITs to a stock or bond portfolio will increase the Sharpe ratio.
        \item The diversification benefit of direct investment makes up for the higher returns from REIT investment causing similar Sharpe ratios.
    \end{itemize}
\end{flashcard}

\begin{flashcard}[\studyArea]{Effects of Adding Private Equity to an Existing Portfolio}
    \begin{itemize}
        \item Increases long-term returns more than diversifies.
        \item Usually illiquid, require a long-term commitment, have high risk with potential for complete loss, and often have a minority discount.
        \item VC firms have higher risk than buyout funds because of differences in information transparency. This can increase the return to VC investors.
        \item Private equity generally moves with the stock market, with low, positive correlation.
    \end{itemize}
\end{flashcard}

\begin{flashcard}[\studyArea]{Effects of Adding Commodities to an Existing Portfolio}
    \begin{itemize}
        \item Mainly offer diversification to stocks and bonds. Correlations are low or slightly negative.
        \item Except for agricultural, strong positive correlation with inflation and provide a hedge against inflation.
        \item Returns are generally lower than stocks and bonds, with the energy subgroup performing best.
        \item High volatility of commodities can give it a low Sharpe ratio.
    \end{itemize}
\end{flashcard}

\begin{flashcard}[\studyArea]{Effects of Adding Hedge Funds to an Existing Portfolio}
    \begin{itemize}
        \item Vary widely, so the benefits in individual styles will differ.
        \item From 1990 to 2004, had higher absolute and risk-adjusted returns, but faired less well from 200 to 2004.
    \end{itemize}
\end{flashcard}

\begin{flashcard}[\studyArea]{Effects of Adding Managed Futures to an Existing Portfolio}
    \begin{itemize}
        \item Considered a category of hedge fund with similar performance. Usually compared to stocks and bonds.
        \item Similar return to stocks with a higher Sharpe ratio. Higher returns than bonds with a lower Sharpe ratio.
        \item Evidence to suggest there are excess returns produced by active management strategies. This can be seen from the low correlation between an index of separately managed accounts and a stock and bond portfolio.
    \end{itemize}
\end{flashcard}

\begin{flashcard}[\studyArea]{Effects of Adding Distressed Securities to an Existing Portfolio}
    \begin{itemize}
        \item Relatively high return with a large negative skew means that comparisons using averages and Sharpe ratios can be misleading.
        \item High returns because many investors cannot use them and there are few analysts covering the market.
        \item Returns are often even-driven, so there is low correlation with the stock market.
    \end{itemize}
\end{flashcard}

\begin{flashcard}[\studyArea]{Advantages and Disadvantages to Direct Real Estate Investment}
    \begin{itemize}
        \item Advantages
        \begin{itemize}
            \item Many expenses are tax deductible.
            \item Ability to use more leverage then usual.
            \item Direct control of properties.
            \item Ability to diversify geographically.
            \item Lower volatility of returns than stocks.
        \end{itemize}
        \item Disadvantages
        \begin{itemize}
            \item Lack of divisibility.
            \item high information cost.
            \item High commissions.
            \item High operating and maintenance costs and management requirements.
            \item Geographic risks, such as neighborhood deterioration.
            \item Political risks, such as changing tax codes.
        \end{itemize}
    \end{itemize}
\end{flashcard}

\begin{flashcard}[\studyArea]{Types of Venture Capital Investors}
    \begin{itemize}
        \item \textbf{Venture capitalists} identify pools of available investment capital and find promising private companies to invest in.
        \item \textbf{Corporate venturing} is when large companies take part in VC opportunities in their own business area.
        \item \textbf{Angel investors} are often the first outsiders who invest in the company.
    \end{itemize}
\end{flashcard}

\begin{flashcard}[\studyArea]{Stages of Venture Capitalized Companies}
    \begin{itemize}
        \item \textbf{Early stage.} Includes seed money often put up by the founders, start-up funds to begin product development, and first-stage funding to begin manufacturing and sales.
        \item \textbf{Expansion state.} Can include young companies with a well-established product, more established companies looking for growth, or those looking to IPO. Second-stage financing supports further expansion. Mezzanine or bridge financing is used to prepare for an IPO and many include both debt and equity capital.
        \item \textbf{Exit stage.} May involve an IPO, merger, or acquisition (which may be another VC firm).
    \end{itemize}
\end{flashcard}

\begin{flashcard}[\studyArea]{Differences Between Venture Capital and Buyout Funds}
    \begin{flushleft}
        In contrast to VC funds, buyout funds usually have
        \begin{itemize}
            \item Higher leverage.
            \item Earlier and steadier cash flows.
            \item Less error in return measurement and more return made up of cash flows.
            \item Less frequent losses.
            \item Less upside potential.
        \end{itemize}
    \end{flushleft}
\end{flashcard}

\begin{flashcard}[\studyArea]{Use of Convertible Preferred Stock in Venture Capital Investment}
    \begin{flushleft}
        Good vehicle for direct VC investment because stockholders must be paid a specified amount before common shareholders can receive dividends.\newline

        Any buyout of the company that is favorable to shareholders will lead to the conversion of preferred stock.\newline

        Usually subsequent rounds of financing have preferred stock with a higher claim than preceding rounds. This is to entice later investors as these shares are more valuable than earlier shares. 
    \end{flushleft}
\end{flashcard}

\begin{flashcard}[\studyArea]{Private Equity Funds}
    \begin{flushleft}
        Generally are formed in limited partnerships or limited liability companies (LLCs). For LPs, the sponsor is called the general partner, for LLCs, the sponsor is called the managing director. The sponsor constructs and manages the fund.\newline

        Funds start with commitments from investors and then giving capital calls over the next five years. This is the commitment period. The expected life of a fund is seven to ten years.\newline

        Sponsors receive compensation through either capital return or management and incentive fees. Management fees are usually 1.5\% to 2.5\% and are based on committed funds.\newline

        Incentive fees, or carried interest, are the share of the profits, usually 20\%, paid after return of committed capital. Incentive fees are often not paid until a hurdle rate has been paid to investors. A sponsor may receive early distribution, but there may be a claw-back provision on the money.
    \end{flushleft}
\end{flashcard}

\begin{flashcard}[\studyArea]{Concerns for Private Equity Investment Strategies}
    \begin{itemize}
        \item \textbf{Low liquidity.} Portfolio allocation should be 5\% or less, with a plan to hold for seven to ten years.
        \item \textbf{Diversification through multiple positions.} Commitments are large, so investors should have \$100 million to invest in five to ten investments.
        \item \textbf{Diversification strategy.} Know the unique aspects of an investment as it relates to the overall portfolio.
        \item \textbf{Plans for meeting capital calls.} Committed funds are called as needed, so the investor needs to be prepared to meet them.
    \end{itemize}
\end{flashcard}

\begin{flashcard}[\studyArea]{Direct Versus Indirect Commodity Investment}
    \begin{itemize}
        \item Direct investment gives more exposure, but cash investments can incur carrying costs.
        \item Indirect investment may be convenient, but can have little exposure, particularly if the company is hedging the risk itself.
        \item Derivatives have the best advantage and make commodity investing available to smaller investors.
    \end{itemize}
\end{flashcard}

\begin{flashcard}[\studyArea]{Three Components of Return for a Commodity Futures Contract}
    \begin{flushleft}
        The three components are additive, so
        \[
            \text{total return} = \text{spot return} + \text{collateral return} + \text{roll return}
        \]
        \begin{enumerate}
            \item \textbf{Spot return} or \textbf{price return} of the underlying commodity. Can be both positive or negative.
            \item \textbf{Collateral return} is the periodic risk-free return. There is an implicit assumption that cash equivalents equal to the full price of the contract are held. Will be positive.
            \item \textbf{Roll yield} is the change in the futures price for the time period, minus the change in the spot price of the commodity. It can be positive or negative.
        \end{enumerate}
        Roll yield is affected by the term structure of futures contracts.
        \begin{itemize}
            \item \textbf{Backwardation} is a downward-sloping term structure of futures prices (i.e., each successive futures price is lower). This predicts a positive roll yield as futures prices increase to converge with the spot price.
            \item \textbf{Contago} is an upward-sloping futures term structure. This predicts a negative roll yield.
        \end{itemize}
    \end{flushleft}
\end{flashcard}

\begin{flashcard}[\studyArea]{Commodities as an Inflation Hedge}
    \begin{flushleft}
        The two factors determining whether a commodity is a good inflation hedge are storability and demand relative to economic activity.\newline

        Storable commodities (e.g., gold, silver, zinc, copper, crude oil or natural gas) are positively related to unexpected changes in inflation. They have provided good hedges against inflation. Non-storable commodities like agricultural commodities tend to be negatively affected by unexpected increases in inflation and as such are not good inflation hedges.\newline

        Commodities that have a more constant demand regardless of economic activity are generally poor inflation hedges. Agricultural commodities are characterized by this. Commodities which are affected by level of economic activity (e.g., precious metals or energy) tend to be better hedges.
    \end{flushleft}
\end{flashcard}

\begin{flashcard}[\studyArea]{Classifications of Hedge Funds by Strategy}
    \begin{itemize}[itemsep=.5\itemsep]
        \item \textbf{Convertible arbitrage} exploits mispricings of convertibles such as convertible bonds, preferred stock or warrants. Hedge risk with long and short positions.
        \item \textbf{Distressed securities} may be undervalued and offer better returns because many investors are not allowed to invest in them. Generally illiquid making shorting difficult and so funds are typically long and not hedged.
        \item \textbf{Emerging markets} generally only use long positions, often with no derivatives.
        \item \textbf{Equity market neutral} combines long and short positions in under- and over-valued securities to eliminate systematic risk.
        \item \textbf{Hedge equity strategies} have long and short positions in under- and over-valued securities to exploit mispricings. Unlike market neutral funds, they don't remove systematic risk. Might be long, short or neutral depending on markets.
        \item \textbf{Fixed income arbitrage} involves taking long and short fixed income positions based on expected changes in the yield or credit curves.
        \item \textbf{Global macro strategies} take positions in major financial and non-financial markets. Focus on an entire investment area instead of individual securities or classes.
        \item \textbf{Merger arbitrage} focuses on mergers, spin-offs, takeovers, etc.
        \item \textbf{Fund of funds} is a single fund which invests in many hedge funds. Has higher fees as all managers get paid.
    \end{itemize}
\end{flashcard}

\begin{flashcard}[\studyArea]{Classifications of Hedge Funds by Segment}
    \begin{itemize}
        \item \textbf{Relative value} strategies exploit mispricings. Includes equity market neutral, convertible arbitrage, and fixed income arbitrage.
        \item \textbf{Event-driven} strategies invest short-term based on the outcome of an event. Includes merge arbitrage and distressed securities.
        \item \textbf{Equity hedge} involves taking long and short equity positions with varying overall net long or short positions and can include leverage.
        \item \textbf{Global asset allocators} take long and short positions in both financial and non-financial assets.
        \item \textbf{Short selling} takes short-only positions with the expectation of a decline in value.
    \end{itemize}
\end{flashcard}

\begin{flashcard}[\studyArea]{Fee Structure of Hedge Funds}
    \begin{flushleft}
        The most common fee structure includes an assets under management fee of 1\% to 2\% and an incentive fee of 20\% of the profits.\newline

        High water marks are used to avoid incentive fee double-dipping. They prevent a manager from earning an incentive fee if the value of the fund is not higher than the highest previous value. High water marks are investor- and date-specific. That is, if an investor commits funds during a poor period, any previous high water marks are not relevant.\newline

        Many funds have lock-up periods which limit withdrawals by requiring a minimum investment period. The idea is to prevent sudden withdrawals causing managers to unwind positions.
    \end{flushleft}
\end{flashcard}

\begin{flashcard}[\studyArea]{Fund of Funds Hedge Funds}
    \begin{flushleft}
        Hedge fund that consists of usually 10 to 30 hedge funds. The idea is to add diversification, but there are extra management fees. Often provides diversification. Good entry-level investments.\newline

        Can be a better indicator of aggregate hedge fund performance than hedge fund indices because it doesn't suffer from backfill or survivorship biases. If a fund performs poorly or is eliminated, those returns are reflected in the fund of funds returns.\newline

        Can suffer from style drift, which may cause an investor to be unclear about what he is invested in. May also cause two fund of funds claiming the same style to have low correlation.\newline

        Have higher correlation with equity markets than individual hedge funds, which causes them to be used as equity diversifiers.
    \end{flushleft}
\end{flashcard}

\begin{flashcard}[\studyArea]{Issues Concerning Hedge Fund Performance Evaluation}
    \begin{itemize}
        \item \textbf{Absolute return vehicles} are those which have no direct benchmark to compare to. Hedge funds target a specific return per period, but this isn't a good benchmark because it isn't investable. Possible solutions to this are single- or multi-factor models or tracking portfolios with comparable risk and return characteristics.
        \item \textbf{Conventions} specific to hedge funds affect performance. This results in
        \begin{itemize}
            \item Funds with longer lock-up periods produce higher returns.
            \item Younger funds tend to outperform older funds.
            \item Large funds underperform smaller funds.
        \end{itemize}
        \item \textbf{Returns} are computed as $V_1 / V_0 - 1$ monthly and then compounded. To smooth out reported returns, use a 12-month rolling average which will always include 12 months of returns.
        \item \textbf{Leverage} is handled by assuming all debt is paid off. This still applies when derivatives are involved. 
        \item \textbf{Risk} measured using standard deviation can give misleading results. This is because hedge fund returns are usually skewed with significant leptokurtosis so standard deviation doesn't measure the true risk of the distribution.
    \end{itemize}
\end{flashcard}

\begin{flashcard}[\studyArea]{Downside Deviation}
    \begin{flushleft}
        Downside deviation measures only the dispersion of returns below some specified threshold return. It is calculated as
        \[
            \text{downside deviation} = \sqrt{\frac{\sum_{t=1}^n \min(\text{return}_t - \text{threshold}, 0)^2}{n - 1}}
        \]
        The threshold is usually either zero or the risk-free rate. It it's a recent average return, then the downside deviation is called semivariance.\newline

        The idea behind downside deviation is to focus on negative returns and not penalize a fund for high positive returns.
    \end{flushleft}
\end{flashcard}

\begin{flashcard}[\studyArea]{Limitations of Using Sharpe Ratio for Hedge Funds}
    \begin{flushleft}
        Hedge fund Sharpe ratios are computed using annualized returns.
        \[
            \text{Sharpe}_{\text{HF}} = \frac{\text{annualized return} - \text{annualized risk-free rate}}{\text{annualized standard deviation}}
        \]
        Disadvantages to hedge fund Sharpe ratios are
        \begin{itemize}
            \item \textbf{Time dependency.} The annual ratio is typically estimated using shorter time periods.
            \item \textbf{Assumes normality.} The ratio uses standard deviation, which is inappropriate for skewed return distributions.
            \item \textbf{Assumes liquidity.} Infrequent or missing returns data points causes downward-biased standard deviations.
            \item \textbf{Assumes uncorrelated returns.} Returns correlated across time will artificially lower standard deviation.
            \item \textbf{Stand-alone measure.} Does not consider diversification effects.
            \item \textbf{Uses historical data.} Has little power to predict winners.
            \item \textbf{Can be manipulated.} Managers can report higher returns to inflate the ratio.
        \end{itemize}
    \end{flushleft}
\end{flashcard}

\begin{flashcard}[\studyArea]{Trading Strategies of Managed Futures Programs}
    \begin{flushleft}
        Managed futures are run by Commodity Pool Operators (CPOs), which themselves can be commodity trading advisors (CTAs). Typically classified by style or strategy. Sometimes considered a subset of global macro hedge funds specializing in derivatives.\newline

        CTA strategies can be classified two different ways.
        \begin{itemize}
            \item \textbf{Systematic trading strategies} apply sets of rules to trade along short-, intermediate-, or long-term trends. They may also trade in a contrarian strategy.
            \item \textbf{Discretionary trading strategies} are based on the discretion of the CTA.
        \end{itemize}

        Managed futures can also be classified by the markets in which they trade. Those which trade in
        \begin{itemize}
            \item \textbf{Financial markets} trade in interest rate and currency futures, options and forward contracts.
            \item \textbf{Currency markets} trade exclusively in currency derivatives.
            \item \textbf{Diversified markest} trade in all financial derivatives markets above as well as commodity derivatives.
        \end{itemize}
    \end{flushleft}
\end{flashcard}

\begin{flashcard}[\studyArea]{Managed Futures Role in a Portfolio}
    \begin{flushleft}
        The primary benefit to managed futures is the significant diversification offered. Private funds seem to add value whereas publicly traded funds have performed poorly, both stand-alone and in portfolios.\newline

        Risk should be considered when selecting a CTA. The beta corresponding to an individual CTA compared to a fund of CTAs can be a good indicator of risk-adjusted performance. This works similar to stock betas in comparison to the overall market.
    \end{flushleft}
\end{flashcard}

\begin{flashcard}[\studyArea]{Strategies for Distressed Securities Investing}
    \begin{itemize}
        \item \textbf{Long-only value investing} finds opportunities where prospects will improve.
            \begin{itemize}
                \item \textit{High-yield investing} buys publicly-traded, below-investment grade debt.
                \item \textit{Orphan equities investing} purchases equities of firms emerging from reorganization.
            \end{itemize}
        \item \textbf{Distressed debt arbitrage} purchases a company's distressed debt while shorting the equity. Can earn a return in two ways.
            \begin{enumerate}
                \item If conditions worsen, debt and equity will both fall in value, but equity will fall more because debt has seniority.
                \item If conditions improve, the priority of interest payments over dividend payments means debtholders will have higher returns.
            \end{enumerate}
        \item \textbf{Private equity} acquires positions in the distressed company which give some amount of control. The investor can then assist and gain more ownership during reorganization. Vulture funds specialize in purchasing undervalued distressed securities use this strategy.
    \end{itemize}
\end{flashcard}

\begin{flashcard}[\studyArea]{Risks Associated with Distressed Securities Investing}
    \begin{itemize}
        \item \textbf{Event risk} refers to the idea that return on investment depends on an event for a particular company. These events are generally unrelated to the economy and provide diversification benefits.
        \item \textbf{Market liquiditiy risk} refers to low liquidity and the possible cyclical supply and demand for these investments.
        \item \textbf{Market risk} from macroeconomic changes is usually less important than event or market risk.
        \item \textbf{J-factor risk} is risk presented by courts and judges which is an unpredictable human element. Anticipation of a bankruptcy judge's ruling, an investor knows whether to buy the distressed company's debt or equity.
    \end{itemize}
\end{flashcard}
\end{document}
