\documentclass[../custom,grid]{flashcards}
\usepackage{booktabs}
\usepackage{array}

\begin{document}
\cardfrontstyle{headings}
\cardfrontfoot{Study Session 13}

\begin{flashcard}[Alternative Investments]{Roles of Alternative Investments in Portfolio Management}
    \begin{itemize}
        \item Real estate and long-only commodities give exposure to risk and return that stocks and bonds don't.
        \item Hedge funds and managed futures give exposure to special strategies and are dependent on manager skill.
        \item Private equity and distressed securities are a combination of 1 and 2.
    \end{itemize}
\end{flashcard}

\begin{flashcard}[Alternative Investments]{Common Features of Alternative Investments}
    \begin{itemize}
        \item \textbf{Low liquidity.} Requires attention to determine if they suitable for a given investor. They should have a liquidity premium and higher return.
        \item \textbf{Diversification.} Generally low correlation with stocks and bonds.
        \item \textbf{Due diligence costs.} Costs of researching and monitoring investments can be high. Specialized expertise and skills are often required. Markets often lack transparency, making information difficult to find.
        \item \textbf{Difficult performance evaluation.} Lack of transparency and unique features of strategies make valuation benchmarks difficult to identify.
    \end{itemize}
\end{flashcard}

\begin{flashcard}[Alternative Investments]{Steps for Due Diligence in Alternative Investment Manager Selection}
    \begin{enumerate}
        \item \textit{Asses the market opportunity offered.} Confirm that there are market inefficiencies for the type of investment in which the manager specializes. Past returns don't justify selecting a manager if there are no opportunities for him to exploit.
        \item \textit{Asses the investment process.} Identify the manager's competitive edge over others. How does his process identify investment opportunities.
        \item \textit{Assess the organization.} Is it stable and well-run? What is the staff turnover?
        \item \textit{Assess the people.} Meet and assess their character, including integrity and competence.
        \item \textit{Assess the terms and structure of the investment.} Consider fee structure, lock-out period, exit strategy and how it aligns the interests of the manager with the investors.
        \item \textit{Asses the service providers.} Investigate outside firms supporting the manager's business (e.g. lawyers or brokers).
        \item \textit{Review documents.} Review the prospectus, audits of manager's reports and other documents. Use legal advice when needed.
        \item \textit{Write-up.} Document the above review process.
    \end{enumerate}
\end{flashcard}

\begin{flashcard}[Alternative Investments]{Issues to Consider for Individual Investors}
    \begin{itemize}
        \item \textbf{Taxes.} Many alternative investments are structured as limited partnerships which require specialized tax expertise.
        \item \textbf{Suitability.} Time horizon and liquidity needs must be considered as many alternative investments must stay invested for a minimum time period.
        \item \textbf{Commuunication.} Make sure the client understands the implications of a lock-out period or a complex strategy as much as is needed.
        \item \textbf{Decision risk.} Preparedness for losses and the ability to remain in a position at a maximum loss point.
        \item \textbf{Concentrated positions.} Wealthy individuals often have large positions in closely held companies or private residences. These should be considered as preexisting allocations before adding more private equity or real estate investments. These positions may also have unrealized taxable gains that can complicate rebalancing.
    \end{itemize}
\end{flashcard}

\begin{flashcard}[Alternative Investments]{Characteristics of Real Estate Investment}
    \begin{flushleft}
        Can be classified as direct or indirect. Direct investment includes ownership and management of residences, commercial real estate and agricultural land. Indirect investment generally involves a separate entity to manage the properties. Indirect investments include
        \begin{itemize}
            \item Companies that develop and manage real estate.
            \item Real estate investment trusts, REITs, which are publicly traded shares in a real estate portfolio. Divided into equity and mortgage REITs. Can be purchased in small sizes and are liquid.
            \item Commingled real estate funds, CREFs, which are pooled investments in real estate that are professionally managed and privately held. Are more flexible than REITs, can be either open- or closed-ended. Restricted to wealthy investors and institutions.
            \item Infrastructure funds purchase infrastructure assets (e.g., airports or toll roads). Because they provide a public service and are regulated, tend to have stable, long-term returns. Low correlation with equities. Long-term nature means they are used by institutions with long-term liabilities. Relatively low returns.
        \end{itemize}
        Advantages of real estate investment are low correlation with stocks and bonds, low volatility of returns and an inflation hedge. May offer tax advantages.\newline

        Disadvantages are high information and transaction costs, political risks from tax law changes, high operating expenses, and inability to subdivide investments.
    \end{flushleft}
\end{flashcard}

\begin{flashcard}[Alternative Investments]{Characteristics of Private Equity Investment}
    \begin{flushleft}
        Investment in non-publicly-traded company. Limited to wealthy individuals or institutions. Often done through pooling funds with others in a private equity fund. Two most important categories are venture capital and buyout funds.\newline

        Two segments of buyout funds are middle-market buyout funds and mega-cap buyout funds. Middle-market funds concentrate on divisions of large publicly-traded companies and small private companies that cannot obtain capital. Mega-cap funds concentrate on taking public firms private.\newline

        Buyout funds add value through some combination of
        \begin{enumerate}
            \item Restructuring company operations and management.
            \item Buying companies for less than intrinsic value.
            \item Creating value by added leverage or restructuring existing debt.
        \end{enumerate}
        Exit strategies are selling the company through private placement or IPOs or through dividend recapitalizations. In the latter, the debt-equity structure is changed by issuing lots of debt and then the company pays a large special dividend to the buyout fund. The fund retains ownership but recoups its cash.\newline

        Private equity is diverse and typically very risky. Venture capitalists are expected to bring both capital and business expertise to a company that might lack management skills.
    \end{flushleft}
\end{flashcard}

\begin{flashcard}[Alternative Investments]{Characteristics of Investment in Commodities}
    \begin{flushleft}
        Investment can be direct or indirect.
        \begin{itemize}
            \item Direct investment includes direct purchase of the physical commodity (e.g., agricultural products, crude oil, or metals) or the purchase of derivatives on those assets.
            \item Indirect investment can be investment in companies whose business is tied to a commodity (e.g., investing in metal via shares in a mining company).
            \item Direct investment through derivatives is more common as indirect investment doesn't track price changes and buying outright creates storage costs.
        \end{itemize}
        Commodity futures and publicly traded commodity companies are liquid when compared with other alternative investments. Investments in commodities have low correlation with stocks and bonds, business-cycle sensitivity and generally have positive correlation with inflation.
    \end{flushleft}
\end{flashcard}

\begin{flashcard}[Alternative Investments]{Characteristics of Hedge Fund}
    \begin{flushleft}
        Generally structured to avoid regulation, which allows them to charge large incentive fees. Designed to exploit a perceived market opportunity, often taking long and short positions on a leveraged basis. Many self-describe as exploiting arbitrage opportunities, although this is used loosely to describe low-risk and not risk-free opportunities.
    \end{flushleft}
\end{flashcard}

\begin{flashcard}[Alternative Investments]{Characteristics of Managed Futures Investments}
    \begin{flushleft}
        Similar to hedge funds in that they are skilled-based and have a similar fee structure. Tend to trade only in derivatives and take positions based on indices. Like hedge funds with a macro focus instead of micro.\newline

        Investment can be done through private commodity pools, managed futures programs as separately managed accounts (called CTA managed accounts), and publicly-traded commodity futures funds available to small investors. Liquidity is higher for publicly-traded funds. Trading strategies used include
        \begin{itemize}
            \item Systematic trading strategies follow rules. Trend following rules may focus on short-, medium-, or long-term trends and are common. Contrarian strategies are less common and have higher diversification.
            \item Discretionary trading strategies use manager judgment and use economic or other criteria.
            \item May invest in all financial markets, only currency markets or a diversified mix of derivatives and underlying commodities.
        \end{itemize}
        The risk characteristics very as with hedge funds. Standard deviation is generally less than half of equities, but greater than bonds. The correlation with equities is low and often negative, higher with bonds but less than 0.5.
    \end{flushleft}
\end{flashcard}

\begin{flashcard}[Alternative Investments]{Characteristics of Investment in Distressed Securities}
    \begin{flushleft}
        Securities of companies that are in or near bankruptcy. Risk and return depend on skill-based strategy. Can construct subgroups based on structure.
        \begin{itemize}
            \item If structured like a hedge fund, there is more liquidity.
            \item If structured like private equity, there is less liquidity because of fixed terms and closed-ended funds.
            \item The private equity structure is useful when the underlying assets are too illiquid to determine a NAV.
        \end{itemize}
    \end{flushleft}
\end{flashcard}

\begin{flashcard}[Alternative Investments]{Comparison of Alternative Investment Characteristics}
    \begin{tabular}{
        >{\raggedright}p{.8in}
        >{\raggedright}p{1.2in}
        >{\raggedright}p{1.2in}
        >{\raggedright\arraybackslash}p{1in}}
        \toprule
        & \textit{Types of Investments} & \textit{Risk/Return Features} & \textit{Liquidity}\\ \midrule
        Real estate &
        Residences, commercial real estate, raw land. &
        Large risk component, good diversification. &
        Low.\\ \midrule
        Private equity &
        Preferred shares, VC, buyout funds. &
        Startup and MM have more risk. &
        Low.\\ \midrule
        Buyout funds &
        Large private firms and spin-offs. &
        Less risk than VC, high diversification. &
        Low. \\ \midrule
        Infrastructure funds &
        Public infrastructure assets. &
        Low risk and return, diversification. &
        Low. \\ \midrule
        Commodities &
        Agriculture, oil, metals. &
        Low correlation with stocks, high with inflation. &
        Fairly liquid.\\ \midrule
        Managed futures &
        Generally only derivatives. &
        Risk between stocks and bonds. &
        Lower for private funds.\\ \midrule
        Distressed securities &
        Can be debt or equity. &
        Depends on skill. High returns. &
        Liquid as hedge funds.\\ \bottomrule
    \end{tabular}
\end{flashcard}

\begin{flashcard}[Alternative Investments]{Comparison of Alternative Investment Benchmarks}
    \begin{tabular}{
        >{\raggedright}p{.8in}
        >{\raggedright}p{1in}
        >{\raggedright}p{1.2in}
        >{\raggedright\arraybackslash}p{1.2in}}
        \toprule
        & \textit{Benchmarks} & \textit{Construction} & \textit{Biases}\\ \midrule
        Real estate &
        NCREIF, NAREIT. &
        NACREIF value weighted, NAREIT cap weighted. &
        Volatility is downward biased.\\ \midrule
        Private equity &
        Cambridge Associates and Thomson Venture Economics. &
        Constructed for buyout and VC. Depends on events. Custom benchmarks. &
        Infrequent repricing means dated values.\\ \midrule
        Commodities &
        Dow Jones-UBS and S\&P Commodity Indices. &
        Assume futures-based strategy. &
        Indices vary widely depending on purpose, makeup, and weight.\\ \midrule
        Managed futures &
        MLMI, CTA indices. &
        MLMI uses trend-following, CTA uses dollar weights. &
        Requires special weighting scheme.\\ \midrule
        Distressed securities &
        Similar to long-only hedge fund benchmarks. &
        Either equal or based on AUM. Selection criteria varies. &
        Self-reporting, backfill, inclusion, popularity, survivorship biases.\\ \bottomrule
    \end{tabular}
\end{flashcard}

\begin{flashcard}[Alternative Investments]{Hedge Fund Benchmarks}
    \begin{flushleft}
        Many differences in indices as style classifications vary. Index providers compose indices using the following.
        \begin{itemize}
            \item \textit{Selection criteria} can include assets under management, track record and new investment restrictions.
            \item \textit{Style classification} varies as to how how a fund is classified by style.
            \item \textit{Weighting schemes} are usually equally weighted or based on assets under management.
            \item \textit{Rebalancing rules} must be defined for equally weighted indices and the frequency can be from monthly to annually.
            \item \textit{Investabilty} depends on reporting frequency. Some indices are not investable, but independent firms make modifications to provide investable proxies.
        \end{itemize}
        Some indices list the funds comprising the index, and some do not. Some report monthly and some report daily.
        \begin{itemize}
            \item Providers of daily indices include Hedge Fund Research (HFR), Dow Jones and S\&P. The DJ and S\&P explicitly list funds and use equal weighting.
            \item Monthly indices include CISDM, equally weighted indices covering hedge funds and managed futures, EACM100 Index, an equally weighted index of 100 funds, and various benchmarks for different strategies based on assets under management from Credit Suisse.
        \end{itemize}
    \end{flushleft}
\end{flashcard}

\begin{flashcard}[Alternative Investments]{Hedge Fund Benchmark Biases}
    \begin{itemize}
        \item \textbf{Relevance of past data} may be questionable. Hedge funds are based on manager skill, and past returns may be a reflection of that skill which may not be present today. Funds with a particular style do have similar returns, and individual managers don't beat their style group.
        \item \textbf{Popularity bias} can occur when one fund attracts a lot of capital in a value weighted index. This overweights its presence in the index, without being caused by return on investment.
        \item \textbf{Survivorship bias} occurs when indices drop funds with poor performance, causing an upward bias in reported values. It can be as high as 1.5-3\% per year. The bias is lower for event-drive strategies and higher for hedged equity strategies.
        \item \textbf{Stale price bias} varies depending on the markets used. When a fund uses markets with infrequent trading, pricing and appraisal can lag causing understatement of volatility. Evidences suggests this is not a large problem.
        \item \textbf{Backfill or inclusion bias} arises from filling in past data. Managers will often only supply data if it benefits them, causing an upward bias in past returns.
    \end{itemize}
\end{flashcard}

\begin{flashcard}[Alternative Investments]{Effects of Adding Real Estate to an Existing Portfolio}
    \begin{itemize}
        \item  High risk-adjusted performance is possible because of low liquidity, large lot sizes, immobility, high transaction costs, and low information transparency.
        \item Adding direct real estate or REITs to a stock or bond portfolio will increase the Sharpe ratio.
        \item The diversification benefit of direct investment makes up for the higher returns from REIT investment causing similar Sharpe ratios.
    \end{itemize}
\end{flashcard}

\begin{flashcard}[Alternative Investments]{Effects of Adding Private Equity to an Existing Portfolio}
    \begin{itemize}
        \item Increases long-term returns more than diversifies.
        \item Usually illiquid, require a long-term commitment, have high risk with potential for complete loss, and often have a minority discount.
        \item VC firms have higher risk than buyout funds because of differences in information transparency. This can increase the return to VC investors.
        \item Private equity generally moves with the stock market, with low, positive correlation.
    \end{itemize}
\end{flashcard}

\begin{flashcard}[Alternative Investments]{Effects of Adding Commodities to an Existing Portfolio}
    \begin{itemize}
        \item Mainly offer diversification to stocks and bonds. Correlations are low or slightly negative.
        \item Except for agricultural, strong positive correlation with inflation and provide a hedge against inflation.
        \item Returns are generally lower than stocks and bonds, with the energy subgroup performing best.
        \item High volatility of commodities can give it a low Sharpe ratio.
    \end{itemize}
\end{flashcard}

\begin{flashcard}[Alternative Investments]{Effects of Adding Hedge Funds to an Existing Portfolio}
    \begin{itemize}
        \item Vary widely, so the benefits in individual styles will differ.
        \item From 1990 to 2004, had higher absolute and risk-adjusted returns, but faired less well from 200 to 2004.
    \end{itemize}
\end{flashcard}

\begin{flashcard}[Alternative Investments]{Effects of Adding Managed Futures to an Existing Portfolio}
    \begin{itemize}
        \item 
    \end{itemize}
\end{flashcard}

\end{document}
