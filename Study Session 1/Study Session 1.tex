\documentclass[../custom]{flashcards}
\usepackage{enumerate}

\newcommand{\studyArea}{Code of Ethics and Standards of Professional Conduct}

\def\labelitemii{$\circ$}
\def\labelitemiii{$\diamond$}
\def\labelitemiv{$\cdot$}

\begin{document}
\cardfrontstyle{headings}
\cardfrontfoot{Study Session 1}

\begin{flashcard}[\studyArea]{Code of Ethics}
    \begin{itemize}
        \item Act with integrity, competence, diligence, respect, and in an ethical manner.
        \item Place the integrity of the investment profession and the interests of clients above your own interests.
        \item Use reasonable care and judgment with analysis, recommendations and investment actions.
        \item Practice and encourage others to practice in an ethical manner.
        \item Promote the integrity of and uphold the rules governing capital markets.
        \item Maintain and improve your professional competence and strive to do so for other investment professionals.
    \end{itemize}
\end{flashcard}

\begin{flashcard}[\studyArea]{Standards of Professional Conduct}
    \begin{enumerate}[I.]
        \item Professionalism
        \item Integrity of Capital Markets
        \item Duties to Clients
        \item Duties to Employers
        \item Investment Analysis, Recommendations, and Actions
        \item Conflicts of Interest
        \item Responsibilities as a CFA Institute Member or CFA Candidate
    \end{enumerate}
\end{flashcard}

\begin{flashcard}[\studyArea]{Professionalism Subsections}
    \begin{enumerate}[A.]
        \item \textbf{Knowledge of the Law.} Understand and comply with all applicable laws. Follow the more strict law or regulation. Must not knowingly participate or assist in and must disassociate from violations.
        \item \textbf{Independence and Objectivity.} Use reasonable care and judgment to maintain independence and objectivity. No gifts or benefits that can compromise their independence or objectivity.
        \item \textbf{Misrepresentation.} No misrepresentations related to analysis, recommendations, actions or other professional activities.
        \item \textbf{Misconduct.} No professional conduct involving dishonesty, fraud or deceit.
    \end{enumerate}
\end{flashcard}

\begin{flashcard}[\studyArea]{Integrity of Capital Markets Subsections}
    \begin{enumerate}[A.]
        \item \textbf{Material Nonpublic Information.} Nonpublic information that could affect the value of an investment cannot be acted upon. Cannot cause others to act upon such information.
        \item \textbf{Market Manipulation.} Cannot take action to distort prices or artificially inflate trading volume with the intent to mislead market participants.
    \end{enumerate}
\end{flashcard}

\begin{flashcard}[\studyArea]{Duties to Clients Subsections}
    \begin{enumerate}[A.]
        \item \textbf{Loyalty, Prucence and Care.} Must act with reasonable care. Place clients' interests before their employer's or their own interests.
        \item \textbf{Fair Dealing.} Deal fairly and objectively with all clients when providing analysis, recommendations or taking investment action.
        \item \textbf{Suitability.}
            \begin{enumerate}[1.]
                \item When in an advisory relationship with a client:
                    \begin{enumerate}[a.]
                        \item Consider the client's investment experience, risk and return, and financial constraints. Must update this information regularly.
                        \item Determine suitability before taking action or making a recommendation.
                        \item Judge the suitability of investments in the context of the entire portfolio.
                    \end{enumerate}
                \item When managing to a specific mandate, strategy or style, recommendations and actions must be consistent with those objectives.
            \end{enumerate}
        \item \textbf{Performance Presentation.} Make reasonable efforts to ensure performance information is fair, accurate and complete.
        \item \textbf{Preservation of Confidentiality.} Must keep information confidential unless:
            \begin{enumerate}[1.]
                \item The information concerns illegal activities.
                \item Disclosure is required by law.
                \item The client or prospective client permits disclosure of the information.
            \end{enumerate}
    \end{enumerate}
\end{flashcard}

\begin{flashcard}[\studyArea]{Duties to Employers Subsections}
    \begin{enumerate}[A.]
        \item \textbf{Loyalty.} Must act for the benefit of their employer, not withhold skills and abilities, or divulge confidential information.
        \item \textbf{Additional Compensation Arrangements.} Must not accept gifts, benefits, compensation or consideration that might create a conflict of interest with their employer's interest unless there's written consent from all parties involved.
        \item \textbf{Responsibilities of Supervisors.} Must make reasonable effort to detect and prevent violations of applicable laws, rules, regulations, and the Code and Standards by anyone subject to their supervision.
    \end{enumerate}
\end{flashcard}

\begin{flashcard}[\studyArea]{Investment Analysis, Recommendations, and Actions Subsections}
    \begin{enumerate}[A.]
        \item \textbf{Diligence and Reasonable Basis.} Members and Candidates must:
            \begin{enumerate}[1.]
                \item Use diligence, independence, and thoroughness in analysis, recommendations, and investment actions.
                \item Support analysis, recommendation and action with research and investigation.
            \end{enumerate}
        \item \textbf{Communications with Clients and Prospective Clients.} Members and Candidates must:
            \begin{enumerate}[1.]
                \item Disclose to clients basic investment processes and any changes that materially affect those processes.
                \item Use judgment identifying investment factors and include those factors in communications with clients.
                \item Distinguish between fact and opinion in presentation of analysis and recommendation.
            \end{enumerate}
        \item \textbf{Record Retention.} Maintain appropriate records to support their analysis, recommendations, actions, and other communications with clients and prospective clients.
    \end{enumerate}
\end{flashcard}

\begin{flashcard}[\studyArea]{Conflicts of Interest Subsections}
    \begin{enumerate}[A.]
        \item \textbf{Disclosure of Conflicts} Must disclose all matters that could impair their independence and objectivity or interfere with duties to clients and employer. Must ensure such disclosures are prominent and delivered in plain language.
        \item \textbf{Priority of Transactions.} Transactions for clients and employers must have priority over transactions in which a Member or Candidate is the beneficial owner.
        \item \textbf{Referral Fees.} Must disclose to their employer, clients and prospective clients, as appropriate, any compensation, consideration or benefit received from or paid to others for the recommendation or products or services.
    \end{enumerate}
\end{flashcard}

\begin{flashcard}[\studyArea]{Responsibilities as a CFA Member or Candidate Subsections}
    \begin{enumerate}[A.]
        \item \textbf{Conduct as Members and Candidates in the CFA Program.} Must not engage in any conduct that compromises the reputation of the CFA Institute, the CFA designation or the CFA exam.
        \item \textbf{Reference to CFA Institute, the CFA Designation, and the CFA Program.} Must not misrepresent or exaggerate the meaning or implications of holding the CFA designation or candidacy in the CFA program.
    \end{enumerate}
\end{flashcard}

\begin{flashcard}[\studyArea]{Knowledge of the Law Recommended Procedures}
    \begin{itemize}
        \item With respect to members
            \begin{itemize}
                \item Have procedures to keep up with changes in the laws, rules, and regulations.
                \item Compliance procedures should regularly to so they address current law, CFAI Standards, and regulations.
                \item Maintain current reference materials for employees to access in order to keep up to date on laws, rules, and regulations.
                \item Seek advice of counsel or their compliance department when in doubt.
                \item Document any violations when disassociating from prohibited activity and encourage employers stop such activity.
                \item Reporting violations to authorities is not required, but may be advisable or required by law.
                \item Strongly encouraged to report other members' violations of the Code and Standards.
            \end{itemize}
        \item With respect to firms
            \begin{itemize}
                \item Develop or adopt a code of ethics.
                \item Make information on laws and regulations available to employees.
                \item Set up written procedures for reporting violations of laws, regulations, or company policies.
            \end{itemize}
    \end{itemize}
\end{flashcard}

\begin{flashcard}[\studyArea]{Independence and Objectivity Recommended Procedures}
    \begin{itemize}
        \item Protect the integrity of opinions---make sure they are unbiased.
        \item Create a restricted list and distribute only factual information about companies on the list.
        \item Restrict special cost arrangements---pay for one's own transportation and hotel. Limit use of corporate aircraft to cases in which commercial transportation is not available.
        \item Limit gifts to token items only. Business-related entertainment is okay provided it doesn't influence independence or objectivity. Firms should impose clear value limits on gifts.
        \item Restrict employee investments in IPOs and private placements. Require IPO approval.
        \item Have effective supervisory and review procedures.
        \item Have formal written policies on independent and objectivity of research.
        \item Appoint a compliance officer and give procedures for employee reporting of unethical behavior and violations of applicable regulations.
    \end{itemize}
\end{flashcard}

\begin{flashcard}[\studyArea]{Misrepresentation Recommended Procedures}
    \begin{itemize}
        \item Provide employees who deal with clients or prospects a list of the firm's services and a qualifications.
        \item Employee qualifications should be accurately presented as well.
        \item Maintain records of all materials used to generate reports or other products and properly cite sources.
        \item Information from recognized financial and statistical reporting services need not be cited.
        \item Members should encourage firms to establish procedures for verifying third party information provided to clients.
    \end{itemize}
\end{flashcard}

\begin{flashcard}[\studyArea]{Misconduct Recommended Procedures}
    \begin{itemize}
        \item Develop and adopt a code of ethics to make clear that unethical behavior will not be tolerated.
        \item Give employees a list of potential violations and sanctions, including dismissal.
        \item Check references of potential employees.
    \end{itemize}
\end{flashcard}

\begin{flashcard}[\studyArea]{Material Nonpublic Information Recommended Procedures}
    \begin{itemize}
        \item Try to get the information made public.
        \item Encourage firms to adopt procedures to prevent misuse of material nonpublic information.
        \item Use a ``firewall'' within the firm, with elements including
            \begin{itemize}
                \item Control of interdepartmental communications through the compliance or legal department.
                \item Review employee trades---maintain ``watch,'' ``restricted,'' and ``rumor'' lists.
                \item Monitor and restrict proprietary trading while a firm has material nonpublic information.

                Stopping all proprietary trading while in possession of nonpublic information may be inappropriate if it sends a signal to the market. In these cases, firms should take the contra side of only unsolicited customer trades.
            \end{itemize}
    \end{itemize}
\end{flashcard}

\begin{flashcard}[\studyArea]{Loyalty, Prudence, and Care Recommended Procedures}
    \begin{itemize}
        \item Submit to clients, at least quarterly, itemized statements.
        \item Encourage firms to address these topics when drafting policies and procedures regarding fiduciary duty.
            \begin{itemize}
                \item Follow applicable rules and laws.
                \item Establish investment objectives of client. Consider suitability of portfolio relative to client's needs, the investment's characteristics, and the characteristics of the total portfolio.
                \item Diversify.
                \item Deal fairly with all clients in regards to investment actions.
                \item Disclose conflicts.
                \item Disclose compensation arrangements.
                \item Vote proxies in the best interest of clients and ultimate beneficiaries.
                \item Maintain confidentiality.
                \item Seek best execution.
                \item Place client interest first
            \end{itemize}
    \end{itemize}
\end{flashcard}

\begin{flashcard}[\studyArea]{Fair Dealing Recommended Procedures}
    \begin{itemize}
        \item Encourage firms to set up procedures for proper dissemination of recommendations and fair treatment of all customers and clients.
        \item Consider these points when establishing fair dealing compliance procedures.
            \begin{itemize}
                \item Limit the people who know about future changes in recommendations.
                \item Shorten the time frame between decision and dissemination.
                \item Use guidelines for pre-dissemination prohibiting personnel who have knowledge of a recommendation from discussing or taking action on it.
                \item Simultaneous dissemination of updated recommendations to everyone who has expressed an interest or for whom an investment is suitable.
                \item Keep a list of clients and holdings to ensure that all holders are treated fairly.
                \item Disclose trade allocation procedures.
                \item Use systematic account review to ensure that no one is given preferred treatment.
                \item Disclose available levels of service.
            \end{itemize}
    \end{itemize}
\end{flashcard}

\begin{flashcard}[\studyArea]{Suitability Recommended Procedures}
    \begin{itemize}
        \item Put the needs, circumstances, and investment objectives of each client into a written IPS.
        \item Consider the type of client and whether there are separate beneficiaries, objectives---return and risk---constraints---liquidity, time, tax, and regulatory and legal circumstances---and performance measurement benchmarks.
        \item Review investor's objectives and constraints periodically to reflect any changes in circumstances.
    \end{itemize}
\end{flashcard}

\begin{flashcard}[\studyArea]{Performance Presentation Recommended Procedures}
    \begin{itemize}
        \item Encourage firms to adhere to GIPS.
        \item Consider the sophistication of the audience during a performance presentation.
        \item Present performance of weighted composite of similar portfolios, not a single account.
        \item Include terminated accounts in historical performance and clearly state termination date.
        \item Include disclosures to fully explain results (e.g., model results included, gross or net of fees, etc.).
        \item Maintain data and records used to calculate the performance being presented.
    \end{itemize}
\end{flashcard}

\begin{flashcard}[\studyArea]{Preservation of Confidentiality Recommended Procedures}
    \begin{itemize}
        \item Avoid disclosing client information except to authorized co-workers who also work for the client.
        \item Follow firm procedures for electronic data storage, and recommend adopting such procedures.
    \end{itemize}
\end{flashcard}

\begin{flashcard}[\studyArea]{Additional Compensation Arrangements Recommended Procedures}
    \begin{itemize}
        \item Make an immediate written report to employer detailing any proposed additional compensation and services. Details, including any performance incentives, should be verified by the offering party.
    \end{itemize}
\end{flashcard}

\begin{flashcard}[\studyArea]{Responsibility of Supervisors Recommended Procedures}
    \begin{itemize}
        \item Recommend employer adopt a code of ethics and provide it to clients.
        \item Employers should not mix compliance procedures with code of ethics.
            \begin{itemize}
                \item Adequate compliance procedures should
                    \begin{itemize}
                        \item Be clearly written and easy to understand.
                        \item Designate a compliance officer with authority clearly defined.
                        \item Have a system of checks and balances.
                        \item Outline the scope of procedures and what conduct is permitted.
                        \item Have procedures for reporting violations and sanctions.
                    \end{itemize}
                \item Once the compliance program in instituted, the supervisor should
                    \begin{itemize}
                        \item Distribute it to the proper personnel.
                        \item Update it as needed, educate staff on procedures, and issue reminders.
                        \item Require professional conduct evaluations.
                        \item Review employee actions to monitor compliance and find violations.
                        \item Enforce procedures once a violation occurs.
                    \end{itemize}
            \end{itemize}
        \item If there is a violation, respond promptly and conduct a thorough investigation while placing limitations on the wrongdoer's activities.
    \end{itemize}
\end{flashcard}

\begin{flashcard}[\studyArea]{Diligence and Reasonable Basis Recommended Procedures}
    \begin{itemize}
        \item Encourage firms to consider these policies and procedures.
            \begin{itemize}
                \item Require research reports and recommendations have a reasonable and adequate basis.
                \item Have detailed, written guidance for proper research and due diligences.
                \item Have measurable criteria for quality of research, and base analyst compensation on that.
                \item Have a minimum acceptable level of scenario testing for computer models, and standards for the range of scenarios, model accuracy, and a measure of the sensitivity of cash flows to model assumptions and inputs.
                \item Have a policy for evaluating third party information providers for reasonableness and accuracy.
                \item Provide criteria for evaluating external advisers and state how often a review of external advisers will be performed.
            \end{itemize}
    \end{itemize}
\end{flashcard}

\begin{flashcard}[\studyArea]{Communication with Clients and Prospective Clients Recommended Procedures}
    \begin{itemize}
        \item Maintain records indicating the nature of research in reports, and be able to supply additional information if it is requested by the client or other users.
    \end{itemize}
\end{flashcard}

\begin{flashcard}[\studyArea]{Record Retention Recommended Procedures}
    \begin{itemize}
        \item This record keeping requirement is generally the firm's responsibility.
    \end{itemize}
\end{flashcard}

\begin{flashcard}[\studyArea]{Disclosure of Conflicts Recommended Procedures}
    \begin{itemize}
        \item Any special compensation arrangements, bonus programs, commissions, and incentives should be disclosed.
    \end{itemize}
\end{flashcard}

\begin{flashcard}[\studyArea]{Priority of Transactions Recommended Procedures}
    \begin{itemize}
        \item Have procedures for conflicts created by personal investing. The following areas should be included.
            \begin{itemize}
                \item Limited or no participation in IPOs.
                \item Strict limits on buying private placements with extra supervisory procedures.
                \item Establish blackout periods prior to trading for clients for employees involved in investment decision-making---no front running. Firm size and security type will indicate severity of blackout requirements.
                \item Establish reporting procedures including duplicate trade confirmations, disclosure of personal holdings or beneficial ownership positions, and preclearance procedures.
                \item Members must fully disclose their firms' personal trading policies to investors when requested.
            \end{itemize}
    \end{itemize}
\end{flashcard}

\begin{flashcard}[\studyArea]{Referral Fees Recommended Procedures}
    \begin{itemize}
        \item Encourage firms to adopt procedures regarding compensation for referrals.
        \item Firms that do not prohibit such fees should have procedures for approval, and members should provide employers with updates at least quarterly with the nature and value of referral compensation received.
    \end{itemize}
\end{flashcard}
\end{document}
