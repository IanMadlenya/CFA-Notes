\documentclass[../custom,grid]{flashcards}
%\usepackage{booktabs}
%\usepackage{array}
%\usepackage{amsmath}
\usepackage{enumitem}

\newcommand{\studyArea}{Risk Management}

\begin{document}
\cardfrontstyle{headings}
\cardfrontfoot{Study Session 14}

\begin{flashcard}[\studyArea]{Five Steps in the Risk Management Process}
    \begin{enumerate}
        \item Setting policies and procedures for risk management.
        \item Defining risk tolerance for various risks based on what the organization's risk profile.
        \item Identifying risks faced by the organization. They can be grouped in financial and non-financial risks. This requires investment databases for both types of risk.
        \item Measuring the current levels of risk.
        \item Adjusting the levels of risk either upward or downward based on the desire to generate returns. These adjustments will involve
        \begin{itemize}
            \item Executing transactions to change the level of risk using derivatives or other instruments.
            \item Finding the most appropriate transaction for a given objective.
            \item Considering the costs of such transactions.
        \end{itemize}
    \end{enumerate}
\end{flashcard}

\begin{flashcard}[\studyArea]{Risk Governance}
    \begin{flushleft}
        Part of the overall corporate governance system and refers to the process of putting a risk management system into use. A good risk governance system will be
        \begin{itemize}
            \item Transparent
            \item Clear in accountability
            \item Cost efficient in the use of resources
            \item Effective in achieving desired outcomes
        \end{itemize}
    \end{flushleft}
\end{flashcard}

\begin{flashcard}[\studyArea]{Centralized Versus Decentralized Risk Governance Systems}
    \begin{itemize}
        \item A decentralized risk governance system puts responsibility for execution with each unit of the organization. The benefit is that risk management is handled by those closest to each part of the organization.
        \item A centralized system, or an enterprise risk management system, puts execution with one central unit. It gives a better view of how risk of each unit affects the risk of the firm as a whole. A centralized system offers economies of scale.
    \end{itemize}
\end{flashcard}

\begin{flashcard}[\studyArea]{Characteristics of an Effective Risk Management System}
    \begin{itemize}
        \item Identifies each risk factor to which the company has exposure.
        \item Quantifies the factor in measurable terms.
        \item Aggregates all risks into a single firm-wide risk metric. VaR is the most common.
        \item Identifies how each risk contributes to the overall firm risk.
        \item Give a process for allocating capital and risk to units of the organization.
        \item Monitor compliance with the allocated limits of capital and risk.
    \end{itemize}
\end{flashcard}

\begin{flashcard}[\studyArea]{Financial Risk Factors}
    \begin{itemize}
        \item \textbf{Market risk} is created by changes in interest rates, exchange rates, market prices, etc. This is frequently the largest component of risk.
        \item \textbf{Credit risk} is the risk of loss caused by a counterparty failing to pay. Historically, credit risk was a binary measurement, but credit derivatives allow for a more continuous measurement. It is often the second largest financial risk.
        \item \textbf{Liquidity risk} is the risk of loss due to the inability to take on or remove a position quickly at a fair price. It can be difficult to measure as liquidity can appear adequate until a certain even occurs. A narrow bid-ask spread generally indicates good liquidity. Average trading volume may give a better indication of liquidity. Liquidity of derivatives is generally linked to that of the underlying security.
    \end{itemize}
\end{flashcard}

\begin{flashcard}[\studyArea]{Non-financial Risk Factors}
    \vspace{-3mm}
    \begin{itemize}[itemsep=.2\itemsep]
        \item \textbf{Operational risk} is loss due to failure of systems or from external events.
        \item \textbf{Settlement risk} is present when funds are exchanged. E.g., if one party makes payment and the other defaults. Risk is low for exchange trades using a clearinghouse. Much higher for OTC transactions.
        \item \textbf{Model risk} refers to the fact that models are only as good as their construction and inputs, (e.g., sensitivities, correlations, likelihoods, etc.).
        \item \textbf{Sovereign risk} is a form of credit risk in which the ability and willingness of a sovereign government must be considered.
        \item \textbf{Regulatory risk} is present when it's unclear how a transaction will be regulated or if that regulation will change.
        \item \textbf{Tax, accounting and legal risk} like regulator risk, refer to situations in which laws may change. Political risk refers specifically to changes in government triggering one of these risks.
        \item \textbf{Environmental, social and governance risk} (ESG) exists if company decisions cause environmental damage, human resource issues, or poor corporate governance which harm the company.
        \item \textbf{Performance netting risk} is when payments from a party are used for another.
        \item \textbf{Settlement netting risk} refers the liquidator of a counterparty in default changing terms of netting agreements such that the non-defaulting party now has to make payments to the defaulting party.
    \end{itemize}
\end{flashcard}

\begin{flashcard}[\studyArea]{Tools for Risk Measurement}
    \begin{itemize}
        \item Standard deviation to measure price or surplus volatility.
        \item Standard deviation of excess return (i.e., the return minus the benchmark return). The standard deviation of excess return is called active risk or tracking risk.
        \item First-order projections of change in price include beta for stocks, duration for bonds and delta for options. Second-order techniques include convexity for bonds and gamma for options.
        \item Option price analysis can also include theta and vega which measure change in price due to change in time to expiration and change in volatility, respectively.
    \end{itemize}
\end{flashcard}

\begin{flashcard}[\studyArea]{Issues to Consider When Using VaR}
    \begin{itemize}
        \item 
    \end{itemize}
\end{flashcard}
\end{document}
