\documentclass[../custom,grid]{flashcards}
%\usepackage{booktabs}
%\usepackage{array}
\usepackage{amsmath}
\usepackage{enumitem}

\newcommand{\studyArea}{Risk Management}

\begin{document}
\cardfrontstyle{headings}
\cardfrontfoot{Study Session 14}

\begin{flashcard}[\studyArea]{Five Steps in the Risk Management Process}
    \begin{enumerate}
        \item Setting policies and procedures for risk management.
        \item Defining risk tolerance for various risks based on what the organization's risk profile.
        \item Identifying risks faced by the organization. They can be grouped in financial and non-financial risks. This requires investment databases for both types of risk.
        \item Measuring the current levels of risk.
        \item Adjusting the levels of risk either upward or downward based on the desire to generate returns. These adjustments will involve
        \begin{itemize}
            \item Executing transactions to change the level of risk using derivatives or other instruments.
            \item Finding the most appropriate transaction for a given objective.
            \item Considering the costs of such transactions.
        \end{itemize}
    \end{enumerate}
\end{flashcard}

\begin{flashcard}[\studyArea]{Risk Governance}
    \begin{flushleft}
        Part of the overall corporate governance system and refers to the process of putting a risk management system into use. A good risk governance system will be
        \begin{itemize}
            \item Transparent
            \item Clear in accountability
            \item Cost efficient in the use of resources
            \item Effective in achieving desired outcomes
        \end{itemize}
    \end{flushleft}
\end{flashcard}

\begin{flashcard}[\studyArea]{Centralized Versus Decentralized Risk Governance Systems}
    \begin{itemize}
        \item A decentralized risk governance system puts responsibility for execution with each unit of the organization. The benefit is that risk management is handled by those closest to each part of the organization.
        \item A centralized system, or an enterprise risk management system, puts execution with one central unit. It gives a better view of how risk of each unit affects the risk of the firm as a whole. A centralized system offers economies of scale.
    \end{itemize}
\end{flashcard}

\begin{flashcard}[\studyArea]{Characteristics of an Effective Risk Management System}
    \begin{itemize}
        \item Identifies each risk factor to which the company has exposure.
        \item Quantifies the factor in measurable terms.
        \item Aggregates all risks into a single firm-wide risk metric. VaR is the most common.
        \item Identifies how each risk contributes to the overall firm risk.
        \item Give a process for allocating capital and risk to units of the organization.
        \item Monitor compliance with the allocated limits of capital and risk.
    \end{itemize}
\end{flashcard}

\begin{flashcard}[\studyArea]{Financial Risk Factors}
    \begin{itemize}
        \item \textbf{Market risk} is created by changes in interest rates, exchange rates, market prices, etc. This is frequently the largest component of risk.
        \item \textbf{Credit risk} is the risk of loss caused by a counterparty failing to pay. Historically, credit risk was a binary measurement, but credit derivatives allow for a more continuous measurement. It is often the second largest financial risk.
        \item \textbf{Liquidity risk} is the risk of loss due to the inability to take on or remove a position quickly at a fair price. It can be difficult to measure as liquidity can appear adequate until a certain even occurs. A narrow bid-ask spread generally indicates good liquidity. Average trading volume may give a better indication of liquidity. Liquidity of derivatives is generally linked to that of the underlying security.
    \end{itemize}
\end{flashcard}

\begin{flashcard}[\studyArea]{Non-financial Risk Factors}
    \vspace{-3mm}
    \begin{itemize}[itemsep=.2\itemsep]
        \item \textbf{Operational risk} is loss due to failure of systems or from external events.
        \item \textbf{Settlement risk} is present when funds are exchanged. E.g., if one party makes payment and the other defaults. Risk is low for exchange trades using a clearinghouse. Much higher for OTC transactions.
        \item \textbf{Model risk} refers to the fact that models are only as good as their construction and inputs, (e.g., sensitivities, correlations, likelihoods, etc.).
        \item \textbf{Sovereign risk} is a form of credit risk in which the ability and willingness of a sovereign government must be considered.
        \item \textbf{Regulatory risk} is present when it's unclear how a transaction will be regulated or if that regulation will change.
        \item \textbf{Tax, accounting and legal risk} like regulator risk, refer to situations in which laws may change. Political risk refers specifically to changes in government triggering one of these risks.
        \item \textbf{Environmental, social and governance risk} (ESG) exists if company decisions cause environmental damage, human resource issues, or poor corporate governance which harm the company.
        \item \textbf{Performance netting risk} is when payments from a party are used for another.
        \item \textbf{Settlement netting risk} refers the liquidator of a counterparty in default changing terms of netting agreements such that the non-defaulting party now has to make payments to the defaulting party.
    \end{itemize}
\end{flashcard}

\begin{flashcard}[\studyArea]{Tools for Risk Measurement}
    \begin{itemize}
        \item Standard deviation to measure price or surplus volatility.
        \item Standard deviation of excess return (i.e., the return minus the benchmark return). The standard deviation of excess return is called active risk or tracking risk.
        \item First-order projections of change in price include beta for stocks, duration for bonds and delta for options. Second-order techniques include convexity for bonds and gamma for options.
        \item Option price analysis can also include theta and vega which measure change in price due to change in time to expiration and change in volatility, respectively.
    \end{itemize}
\end{flashcard}

\begin{flashcard}[\studyArea]{Issues to Consider When Using VaR}
    \begin{itemize}
        \item The VaR time period should relate to the situation. Stocks and bonds might use monthly VaR, while a leveraged derivatives portfolio might use daily VaR.
        \item The percentage selected will affect the VaR. A 1\% VaR will generally show greater risk than a 5\% VaR.
        \item The left tail should be examined. VaR measures the left tail of a distribution, i.e., the negative returns portion. But it only shows a snapshot at a particular percentage.
    \end{itemize}
\end{flashcard}

\begin{flashcard}[\studyArea]{Analytical VaR Method}
    \begin{flushleft}
        Based on the normal distribution and one-tailed confidence intervals.
        \begin{align*}
            \text{VaR} &= \left (\hat{R}_p - z \sigma \right ) V_p\\
            \\
            \text{where:}\\
            \hat{R}_p &= \text{expected return on the portfolio}\\
            V_p &= \text{value of the portfolio}\\
            z &= \text{$z$-value corresponding to the level of significance}\\
            \sigma &= \text{standard deviation of returns}
        \end{align*}
        5\% and 1\% VaR are 1.65 and 2.33 standard deviations below the mean, respectively. For periods less than one year, divide the return and the standard deviation by the number of periods and the square root of the number of periods, respectively. For 1-day VaR, return can be approximated as zero.
    \end{flushleft}
\end{flashcard}

\begin{flashcard}[\studyArea]{Advantages and Disadvantages of the Analytical VaR Method}
    \begin{itemize}
        \item Advantages
        \begin{itemize}
            \item Easy to calculate and understand.
            \item Allows modeling the correlations of risks.
            \item Can be applied to varying time periods as relevant.
        \end{itemize}
        \item Disadvantages
        \begin{itemize}
            \item Assumes normal distribution of returns.
                \begin{itemize}
                    \item Some securities have skewed returns. Long option positions have positive skew with frequent small losses to due premia and occasional large gains. Short option positions have the opposite situations and negative skew.
                    \item Variance-covariance VaR is modified to deal with skew using the delta-normal method. But these adjustments are complex and don't allow convenient second-order modeling.
                \end{itemize}
            \item Many assets have leptokurtosis (i.e., fat tails). VaR tends to underestimate the loss.
            \item Difficult to estimate standard deviation in large portfolios as it requires correlations between assets which grows quadratically with the number of assets.
        \end{itemize}
    \end{itemize}
\end{flashcard}

\begin{flashcard}[\studyArea]{Advantages and Disadvantages of the Historical VaR Method}
    \begin{flushleft}
        Historical VaR or historical simulation takes past daily returns, ranks them and identifies the lowest $N\%$. The highest of these lowest $N\%$ is the 1-day $N\%$ VaR.
        \begin{itemize}
            \item Advantages
            \begin{itemize}
                \item Very easy to calculate and understand.
                \item Does not assume a distribution of returns.
                \item Can be applied to different time periods.
            \end{itemize}
            \item Disadvantages
            \begin{itemize}
                \item Assumes the pattern of historical returns will repeat in the future. Many securities also change characteristics with the passage of time.
            \end{itemize}
        \end{itemize}
    \end{flushleft}
\end{flashcard}

\begin{flashcard}[\studyArea]{Advantages and Disadvantages of the Monte Carlo VaR Method}
    \begin{flushleft}
        The Monte Carlo method generates thousands of possible outcomes from the distributions of inputs specified by the user. The inputted distributions can be normal for some assets, skewed for others, etc. The possible outcomes are then ranked similar to the historical VaR method.
        \begin{itemize}
            \item Advantages
            \begin{itemize}
                \item Can incorporate any assumptions regarding return patterns, correlations, or other factors.
            \end{itemize}
            \item Disadvantages
            \begin{itemize}
                \item The output is only as good as the inputted assumptions.
                \item Its complexity can lead to an overconfidence in the results.
                \item Can be costly and computer intensive to implement.
            \end{itemize}
        \end{itemize}
    \end{flushleft}
\end{flashcard}

\begin{flashcard}[\studyArea]{Advantages and Disadvantages of VaR}
    \begin{itemize}
        \item Advantages
        \begin{itemize}
            \item Industry standard required by many regulators.
            \item Aggregates all risk into a single, easy to understand number.
            \item Can be used in capital allocation by giving each unity a certain amount of VaR. When units have less than perfect correlation, the firm-wide VaR is less than the sum of the unit VaR.
        \end{itemize}
        \item Disadvantages
        \begin{itemize}
            \item Some methods, e.g., Monte Carlo, are difficult and expensive.
            \item Different computation methods can generate different VaR estimates.
            \item Can generate a false sense of security. Only as good as inputs. Additionally, it's a measure of probability, so the situation can always be worse.
            \item One-sided focusing on the downside while ignoring any upside potential.
        \end{itemize}
    \end{itemize}
\end{flashcard}

\begin{flashcard}[\studyArea]{Tools and Metrics to Complement VaR}
    \begin{itemize}
        \item \textbf{Back-testing} should be used to compare actual results with expected outcomes projected by VaR.
        \item \textbf{Incremental VaR} (IVaR) is calculated by measuring the difference in VaR before and after the portfolio is changed in some way (e.g., an asset is added). It measures the effect of an individual item.
        \item \textbf{Cash flow at risk} (CFaR) measures the risk of the company's cash flows. CFaR is useful for companies which cannot be valued directly. CFaR measures the minimum cash flow loss at a given probability over a give time period.
        \item \textbf{Earnings at risk} (EaR) is analogous to CFaR from an accounting earning standpoint.
        \item \textbf{Tail value at risk} (TVaR) is VaR plus the average of the outcomes in the tail.
        \item \textbf{Cerdit VaR} measures risk due to credit events.
        \item \textbf{Stress testing} tests various situations which may occur and their impact on the portfolio.
    \end{itemize}
\end{flashcard}
\end{document}
