\documentclass[../custom,grid]{flashcards}
\usepackage{amsmath}
\usepackage{enumitem}
%\usepackage{enumerate}
%\usepackage{booktabs}
%\usepackage{array}
%\usepackage{tikz}
%\usepackage{float}

\newcommand{\studyArea}{Evaluating Portfolio Performance}

\def\labelitemii{$\circ$}
\def\labelitemiii{$\diamond$}
\def\labelitemiv{$\cdot$}

\begin{document}
\cardfrontstyle{headings}
\cardfrontfoot{Study Session 17}

\begin{flashcard}[\studyArea]{Benefits of Portfolio Evaluation for Fund Sponsors}
    \begin{itemize}
        \item Shows where policy and allocation is and isn't effective.
        \item Directs management to areas of value added and lost.
        \item Quantifies the results of active management.
        \item Shows whether other strategies can be applied successfully.
        \item Provides feedback on the consistent application of polices in the IPS.
    \end{itemize}
\end{flashcard}

\begin{flashcard}[\studyArea]{Three Components of Portfolio Evaluation}
    \begin{enumerate}
        \item \textbf{Performance measurement} calculates rates of return over specified times.
        \item \textbf{Performance attribution} determines the source of the account's performance.
        \item \textbf{Performance appraisal} draws conclusions about the overall issue affecting performance.
    \end{enumerate}
\end{flashcard}

\begin{flashcard}[\studyArea]{Portfolio Return with External Cash Flows}
    \begin{flushleft}
        If there is an external cash flow at the beginning of the period, the return is
        \begin{align*}
            r_t &= \frac{\text{MV}_1 - (\text{MV}_0 + \text{CF})}{\text{MV}_0 + \text{CF}}\\
            \intertext{If there is an external cash flow at the end of the period, it should be subtracted from the account value.}
            r_t &= \frac{(\text{MV}_1 - \text{CF}) - \text{MV}_0}{\text{MV}_0}\\
        \end{align*}
    \end{flushleft}
\end{flashcard}

\begin{flashcard}[\studyArea]{Time- and Money-Weighted Rates of Return}
    \begin{itemize}
        \item \textbf{Time-weighted rate of return} calculates a compounded rate of growth over a stated evaluation period. Subperiod returns are calculated at dates of external cash flows. These returns are then compounded together to get the TWRR.
        \item \textbf{Money-weighted rate of return} is an IRR of all funds invested during the evaluation period, including the beginning value. The MWRR is the value of $R$ such that
            \begin{align*}
                \text{MV}_1 &= \text{MV}_0 (1 + R)^m + \sum_{i=1}^n \text{CF}_i (1 + R)^{L_i}\\
                \text{where:}\\
                \text{MV}_1 &= \text{ending value of the portfolio}\\
                \text{MV}_0 &= \text{beginning value of the portfolio}\\
                m &= \text{number of time units in the evaluation period}\\
                \text{CF}_i &= \text{cash flow $i$}\\
                L_i &= \text{number of time units cash flow $i$ is in the portfolio}
            \end{align*}
    \end{itemize}
\end{flashcard}

\begin{flashcard}[\studyArea]{Comparison Between Time- and Money- Weighted Rates of Return}
    \begin{itemize}
        \item Generally TWRR is used for evaluation and GIPS because it reflects return on assets and not decisions to add or subtract funds.
        \item If the manager controls the timing of the cash flows, MWRR can be used for GIPS reporting.
        \item TWRR shows what would have happened to the portfolio had no cash flows occurred.
        \item TWRR can be data intensive because it requires portfolio market values on dates of all external cash flows.
        \item MWRR requires only a beginning and ending market value.
    \end{itemize}
\end{flashcard}

\begin{flashcard}[\studyArea]{Data Quality Issues in Return Calculations}
    \begin{itemize}
        \item For illiquid assets, price estimates must be used.
        \item Matrix pricing may be used by using dealer-quoted prices on similar securities.
        \item Highly illiquid securities may be carried at cost, not reflecting the current price.
        \item Account valuations should include trade date accounting including accrued interest and dividends.
    \end{itemize}
\end{flashcard}

\begin{flashcard}[\studyArea]{Decomposition of Returns into Market, Style, and Active Management}
    \begin{flushleft}
        \begin{align*}
            P &= M + S + A\\
            \text{where:}\\
            P &= \text{investment manager's portfolio return}\\
            M &= \text{return on market index}\\
            S &= B - M = \text{excess return to style, difference between benchmark and market}\\
            A &= P - B = \text{active return}
        \end{align*}
    \end{flushleft}
\end{flashcard}

\begin{flashcard}[\studyArea]{Properties of a Valid Benchmark}
    \begin{enumerate}
        \item \textbf{Specified in advance.} Known to both the investment manager and the fund sponsor at the start of the evaluation period.
        \item \textbf{Appropriate.} Consistent with manager's approach and style.
        \item \textbf{Measurable.} Value and return can determined on a regular basis.
        \item \textbf{Unambiguous.} Clearly defined identities and weights of components.
        \item \textbf{Reflecive of the manager's current investment opinions.} Manager has current knowledge and expertise of the components.
        \item \textbf{Accountable.} Manager should accept the benchmark and agree to attribute differences to active management.
        \item \textbf{Investable.} Possible to replicate the benchmark and forgo active management.
    \end{enumerate}
\end{flashcard}

\begin{flashcard}[\studyArea]{Seven Types of Benchmarks}
    \begin{itemize}[itemsep=.2\itemsep]
        \item \textbf{Absolute.} A return objective.
        \item \textbf{Manager universes.} Median manager or fund from a universe of such.
        \item \textbf{Broad market indices.} E.g., S\&P 500, MSCI, or EAFE.
        \item \textbf{Style indices.} E.g., large- or small-cap growth or value indices.
        \item \textbf{Factor-model based.} Relate factor exposures to returns.
            \begin{align*}
                R_P &= a_P + \sum_{i=1}^K b_iF_i + \varepsilon\\
                \text{where:}
                R_P &= \text{periodic return on an account}\\
                a_P &= \text{value of $R_P$ if all factors were zero}\\
                F_i &= \text{factors that have systematic effect on performance}\\
                b_i &= \text{sensitivity of the returns to $F_i$}\\
                \varepsilon &= \text{error term, return not explained by model}
            \end{align*}
        \item \textbf{Returns-based.} Constructed using the account's returns and returns on several style indices for the same periods. These are then combined to get an allocation which tracks the account's returns.
        \item \textbf{Custom security-based.} Designed to reflect the manager's security allocations and investment process.
    \end{itemize}
\end{flashcard}

\begin{flashcard}[\studyArea]{Advantages of Different Benchmarks Types}
    \begin{itemize}[nosep]
        \item \textbf{Absolute.}
            \begin{itemize}[nosep]
                \item Simple and straightforward benchmark.
            \end{itemize}
        \item \textbf{Manager universes.}
            \begin{itemize}[nosep]
                \item It is measurable.
            \end{itemize}
        \item \textbf{Broad market indices.}
            \begin{itemize}[nosep]
                \item Well recognized, easy to understand, and widely available.
                \item Unambiguous, investable, and can be specified in advance.
                \item Appropriate if it reflects the manager's approach.
            \end{itemize}
        \item \textbf{Style indices.}
            \begin{itemize}[nosep]
                \item Widely available, understood, and accepted.
                \item Appropriate if it reflects manager's style and is investable.
            \end{itemize}
        \item \textbf{Factor-model based.}
            \begin{itemize}[nosep]
                \item Useful in performance evaluation.
                \item Gives insight into manager's style by showing factor exposures affecting performance.
            \end{itemize}
        \item \textbf{Returns-based.}
            \begin{itemize}[nosep]
                \item Easy to use and intuitive.
                \item Meets the criteria of a valid benchmark.
                \item Useful if the only information available is account returns.
            \end{itemize}
        \item \textbf{Custom security-based.}
            \begin{itemize}[nosep]
                \item Meets the criteria of a valid benchmark.
                \item Allows continual monitoring of investment processes.
                \item Allows sponsors to allocate risk across management teams.
            \end{itemize}
    \end{itemize}
\end{flashcard}

\begin{flashcard}[\studyArea]{Disadvantages of Different Benchmarks Types}
    \begin{itemize}[nosep]
        \item \textbf{Absolute.}
            \begin{itemize}[nosep]
                \item Not an investable alternative.
            \end{itemize}
        \item \textbf{Manager universes.}
            \begin{itemize}[nosep]
                \item Universes are subject to survivorship bias.
                \item Must trust that the universe is accurately compiled.
                \item Cannot be specified in advance.
            \end{itemize}
        \item \textbf{Broad market indices.}
            \begin{itemize}[nosep]
                \item Manager's style may deviate from the index's style.
            \end{itemize}
        \item \textbf{Style indices.}
            \begin{itemize}[nosep]
                \item Style index may contain imprudent weightings.
                \item Different definitions of style can produce different returns.
            \end{itemize}
        \item \textbf{Factor-model based.}
            \begin{itemize}[nosep]
                \item Not intuitive to all managers or sponsors.
                \item Data and modeling may not be available and may be expensive.
                \item Different factor models can produce different output.
            \end{itemize}
        \item \textbf{Returns-based.}
            \begin{itemize}[nosep]
                \item The style indices may not reflect what the manager owns.
                \item A significant number of returns are needed to establish a pattern.
                \item Will not work for managers who change style.
            \end{itemize}
        \item \textbf{Custom security-based.}
            \begin{itemize}[nosep]
                \item Can be expensive to construct and maintain.
                \item Lack of transparency can make it impossible to construct.
            \end{itemize}
    \end{itemize}
\end{flashcard}

\begin{flashcard}[\studyArea]{Steps to Construct a Security-Based Benchmark}
    \begin{enumerate}
        \item Identify the important elements of the manager's process.
        \item Select securities that are consistent with that process.
        \item Weight the securities---including cash---to reflect the manager's process.
        \item Review and adjust as needed to replicate the manager's results.
        \item Rebalance on a predetermined schedule.
    \end{enumerate}
\end{flashcard}

\begin{flashcard}[\studyArea]{Disadvantages to Using Manager Universes as Benchmarks}
    \begin{itemize}
        \item Besides being measurable, it fails all the properties of valid benchmarks.
            \begin{itemize}
                \item Can't identify the median manager in advance.
                \item Because of this, it isn't unambiguous.
                \item Not investable as the median manager will vary between periods.
                \item Impossible to verify appropriateness due to not knowing the median manager.
            \end{itemize}
        \item Must trust the compiler that universe accounts have been screened, data has been validated, and calculation methods approved.
        \item As fund sponsors get rid of underperforming managers, universes are subject to survivorship bias. This biases the median manager upwards.
    \end{itemize}
\end{flashcard}

\begin{flashcard}[\studyArea]{Issues to Consider in Benchmark Evaluation}
    \begin{itemize}
        \item \textbf{Systematic bias.} The historical beta of the account relative to the benchmark should be low. Alternatively, active management returns, $A$, should be uncorrelated with investment style, $S$.
        \item \textbf{Tracking error.} Standard deviation of $A$. Should be smaller than $\sigma (P - M)$. 
        \item \textbf{Risk characteristics.} Account's exposure to systematic sources of risk should be similar to that of the benchmark over time.
        \item \textbf{Coverage.} Coverage ratio is the percentage of portfolio market value that consists of securities in the benchmark. Higher coverage ratio indicates better benchmark replication.
        \item \textbf{Turnover.} Proportion of benchmark's total market value that is bought or sold during a period. Passive portfolios should have benchmarks with low turnover.
        \item \textbf{Positive active positions.} Active position is difference in weights of a security between the portfolio and benchmark. Many negative active positions indicate the benchmark doesn't reflect the manager's process.
    \end{itemize}
\end{flashcard}

\begin{flashcard}[\studyArea]{Issues Arising when Assigning Benchmarks to Hedge Funds}
    \begin{itemize}
        \item Difficult to compute return as many hedge funds hold long and short positions which sum to nearly zero.
        \item Can use a value-added return as the difference between portfolio and benchmark returns.
        \item Some hedge funds target an absolute return and argue benchmarks are irrelevant.
        \item Some funds have a definable style which can be used to compare funds to others of that style. Others do not have a definable style.
        \item Difficulty with benchmarks has lead to using Sharpe ratio, but using standard deviation with hedge funds is questionable.
    \end{itemize}
\end{flashcard}

\begin{flashcard}[\studyArea]{Three Components of Macro Performance Attribution}
    \begin{itemize}
        \item \textbf{Policy allocations.} The sponsor decides the asset categories and weights and allocates funds among managers. These are determined by risk tolerance, long-term expectations, and liabilities the fund must meet.
        \item \textbf{Benchmark portfolio returns.} A sponsor may use broad market indices for asset categories and narrower indices for managers' investment styles.
        \item \textbf{Fund returns, valuations, and external cash flows.} When using percentage terms, returns are calculated on the manager level, allowing the sponsor to compare managers directly. If using monetary terms, more data are needed.
    \end{itemize}
\end{flashcard}

\begin{flashcard}[\studyArea]{Six Levels of Macro Attribution Analysis}
    \begin{enumerate}[nosep]
        \item \textbf{Net contributions} is the net sum of cash inflows or outflows.
        \item \textbf{Risk-free investment} simulates value if only the risk-free return were earned.
        \item \textbf{Asset categories} simulates passive allocation in index funds. Can be calculated as
            \vspace{-3mm}
            \begin{align*}
                R_{\text{AC}} &= \sum_{i=1}^A w_i (R_i - R_F)\\
                %\text{where:}\\
                %R_{\text{AC}} &= \text{incremental return above $R_F$}\\
                %R_i - R_F &= \text{excess return above the risk-free rate for asset category $i$}\\
                %w_i &= \text{weight of asset category $i$}\\
                %A &= \text{number of asset categories}
            \end{align*}
        \item \textbf{Benchmark level} allows manager benchmarks different from the policy benchmark. Can be calculated as
            \vspace{-3mm}
            \begin{align*}
                R_{\text{B}} &= \sum_{i=1}^A \sum_{j=1}^M w_i w_{i,j} (R_{B_{i,j}} - R_i)\\
                %\text{where:}\\
                %R_{\text{B}} &= \text{incremental return for the benchmark strategy}\\
                %w_{i,j} &= \text{weight assigned in manager $j$ in asset category $i$}\\
                %R_{B_{i,j}} &= \text{return for manager $j$'s benchmark in asset category $i$}\\
                %M &= \text{number of managers}
            \end{align*}
        \item \textbf{Active management} simulates returns actually produced by managers. Can be calculated as
            \vspace{-3mm}
            \begin{align*}
                R_{\text{IM}} &= \sum_{i=1}^A \sum_{j=1}^M w_i w_{i,j} (R_{A_{i,j}} - R_{B_{i,j}})\\
                %\text{where:}\\
                %R_{\text{IM}} &= \text{incremental return for the investment manager level}\\
                %R_{A_{i,j}} &= \text{return for manager $j$'s portfolio in asset category $i$}
            \end{align*}
        \item \textbf{Allocation effects} is a plug to sum to the portfolio ending value.
    \end{enumerate}
\end{flashcard}

\begin{flashcard}[\studyArea]{Micro Performance Attribution Formula}
    \begin{flushleft}
        \begin{align*}
            R_V &= \sum_{i=1}^S (w_{P,i} - w_{B,i})(R_{B,i} - R_B) + \sum_{i=1}^S (w_{P,i} - w_{B,i})(R_{P,i} - R_{B,i})\\
            &+ \sum_{i=1}^S w_{B,i} (R_{P,i} - R_{B,i})\\
            \text{where:}\\
            R_V &= \text{value added return}\\
            w_{P,i} &= \text{portfolio weight of sector $i$}\\
            w_{B,i} &= \text{benchmark weight of sector $i$}\\
            R_{P,i} &= \text{portfolio return of sector $i$}\\
            R_{B,i} &= \text{benchmark return of sector $i$}\\
            R_B &= \text{return on the portfolio's benchmark}\\
            S &= \text{number of sectors}
        \end{align*}
    \end{flushleft}
\end{flashcard}
\end{document}
