\documentclass[../custom]{flashcards}
\usepackage{enumitem}

\newcommand{\studyArea}{Equity Portfolio Management}

\def\labelitemii{$\circ$}
\def\labelitemiii{$\diamond$}
\def\labelitemiv{$\cdot$}

\begin{document}
\cardfrontstyle{headings}
\cardfrontfoot{Study Session 12}

\begin{flashcard}[\studyArea]{Breadth Versus Investability}
    \begin{itemize}
        \item \textbf{Breadth} of an index measures its coverage as a percentage of all firms in the market or sector that are in the index. Greater breadth is preferred because the index is a better representation of the market.
        \item \textbf{Investability} is a measure of the index's liquidity. Greater investability is preferred because it lowers costs associated with constructing and rebalancing the portfolio.
    \end{itemize}

    \begin{flushleft}
    With international indices, the shares of small-cap firms can be illiquid. This forces a trade-off between breadth and investability. The greater the breadth, the more illiquid firms the index contains, lowering its investability.
    \end{flushleft}
\end{flashcard}

\begin{flashcard}[\studyArea]{Index Reconstitution}
    \begin{flushleft}
        The process of adding and removing securities from an index. When adding a security, the index generates upward price pressure on that security. Removing generates downward pressure.\newline

        This reconstitution effect is a cost for portfolios tracking the index because they must buy the added securities at a higher price and sell the removed securities at a lower price.
    \end{flushleft}
\end{flashcard}

\begin{flashcard}[\studyArea]{Crossing}
    \begin{flushleft}
        The process used by money managers of matching buy and sell orders from different customers without using a broker and thus incurring no transaction costs.
    \end{flushleft}
\end{flashcard}

\begin{flashcard}[\studyArea]{Precise and Band Float Adjustments}
    \begin{flushleft}
        Some index managers continually adjust the float and resulting market cap weight of stocks in their indices. This is called precise float adjustment and results in frequent rebalancing and high transaction costs.\newline

        Alternatively, other managers use bands and only make adjustments when floats leave those bands.
    \end{flushleft}
\end{flashcard}

\begin{flashcard}[\studyArea]{Objectivity and Transparency Versus Judgment in Index Composition}
    \begin{flushleft}
        Objectivity refers to the use of a fixed set of criteria to decide which securities make up an index. Transparency refers to the availability of said criteria to portfolio managers. These characteristics make it easier for mangers and lower transaction costs.\newline

        Subjective judgment makes index composition less transparent. This is particularly problematic if the index only represents a subset of the industry or market.
    \end{flushleft}
\end{flashcard}

\begin{flashcard}[\studyArea]{Effect of Emerging Market Classifications}
    \begin{itemize}
        \item Countries on the border between emerging and developed, inclusion in an index can effect the index and economy of the country itself.
        \item Inclusion in an emerging market index can cause upward biased results because of its status as a major component of that index.
        \item Inclusion in a developed index can mean the inflow of foreign currency, helping further develop the country.
    \end{itemize}
\end{flashcard}

\begin{flashcard}[\studyArea]{Stakeholders}
    \begin{itemize}[noitemsep]
        \item \textbf{Internal stakeholders}
            \begin{itemize}[noitemsep]
                \item \textbf{Stockholders} supply risk capital to the firm. Risk losing entire investment. Some countries have laws maximizing shareholder interests.
                \item \textbf{Employees} exchange labor for compensation. Generally have the ability to disrupt the company if dissatisfied.
                \item \textbf{Managers} are employees with an asymmetric information advantage in the principal-agent relationship.
                \item \textbf{Board of directors} should monitor the actions of senior managers and act in the interests of shareholders. They also have an asymmetric information advantage and risk becoming too close to managers.
            \end{itemize}
        \item \textbf{External stakeholders}
            \begin{itemize}[noitemsep]
                \item \textbf{Customers} buy products and seek a dependable relationship with the company. They also seek lower prices, which is antithetical to higher profit goals.
                \item \textbf{Suppliers} also seek dependable relationships, but want higher prices which reduce profits.
                \item \textbf{Creditors} are essentially another supplier, but they supply debt capital and are paid interest. They value stable credit quality.
                \item \textbf{Unions} represent internal employees. They often seek higher wages with potential negative implications for short-term profits. Have power to strike.
                \item \textbf{Governments} provide rules and regulations.
                \item \textbf{Local communities} provide infrastructure and expect good citizenship.
                \item \textbf{General public} gives infrastructure in for increased quality of life.
            \end{itemize}
    \end{itemize}
\end{flashcard}

\begin{flashcard}[\studyArea]{Stakeholder Impact Analysis Goals}
    \begin{itemize}
        \item Identify the relevant stakeholders.
        \item Identify the critical interest and desires of each group.
        \item Identify the demands of each group on the company.
        \item Prioritize the importance of various stakeholders to the company.
        \item Provide a business strategy to meet the critical demands.
    \end{itemize}
\end{flashcard}

\begin{flashcard}[\studyArea]{Two Methods to Achieve Stakeholder Satisfaction}
    \begin{flushleft}
        Return on investment capital (ROIC) and earnings growth are commonly used to measure stakeholder demands. These must be maximized within legal and regulatory compliance.\newline

        Excess attention to growth can lead to poor investments and lower ROIC. Focusing too much on ROIC leads to ignoring growth opportunities. There is a middle ground between the two.
    \end{flushleft}
\end{flashcard}

\begin{flashcard}[\studyArea]{Principal-Agent Relationship}
    \begin{flushleft}
        The principal-agent relationship (PAR) is when one group (principal) delegates decision making to another (agent). The PAR is problematic because the agent has an asymmetric information advantage that he may use to his betterment. This is compounded when the information advantage prevents the principal from knowing about the problem.\newline

        To mitigate problems with the PAR, there should be procedures that
        \begin{itemize}
            \item Affect the behavior of agents by setting goals and principles.
            \item Reduce the asymmetry of information.
            \item Remove agents who misbehave and violate ethics.
        \end{itemize}
    \end{flushleft}
\end{flashcard}

\begin{flashcard}[\studyArea]{Sources of Unethical Behavior}
    \begin{itemize}
        \item Agents with flawed personal ethics are more likely to violate business ethics.
        \item Failure to realize an issue may lead to ethics violations. Asking whether each decision is ethical encourages good business ethics.
        \item Culture focused only on profit and growth. Asking if something is ethical and profitable encourages good business ethics.
        \item Flawed business culture where management sets unrealistic goals.
        \item Unethical leadership sets a tone and leads to violations. Ethical leadership forms expectations that include ethical behavior.
    \end{itemize}
\end{flashcard}

\begin{flashcard}[\studyArea]{Philosophies of Ethical Decision Making}
    \begin{itemize}
        \item \textbf{Friedman Doctrine} states that the only social responsibility of businesses is to increase profits within the rules of the game, i.e., through open and fair competition. This doesn't directly address business ethics.
        \item \textbf{Utilitarianism} says businesses must produce the highest good for the largest number of people. Flaws are that costs and benefits are difficult to measure, and the greatest good may exploit a small subgroup.
        \item \textbf{Kantian ethics} argues people are different from other factors of production and require dignity and respect. This is widely accepted, but not sufficient to be a complete philosophy.
        \item \textbf{Rights theories} state that all individuals have fundamental rights and privileges. Managers must have a moral compass recognizing their rights impose an obligation to protect the rights of others.
        \item \textbf{Justice theories} focus on a just distribution of economic output. The rules are fair if all participants, unknowing of the outcome, agree to all the rules.
    \end{itemize}
\end{flashcard}

\begin{flashcard}[\studyArea]{Seven Steps to Further Ethical Behavior of Managers}
    \begin{enumerate}[itemsep=.2\itemsep]
        \item Hire and promote those with strong ethics. Psychological testing and past employment review can help.
        \item Build and organization and culture that values ethical behavior. This includes, stating ethical behavior is required, ethical leadership, and promotion of those who exemplify it.
        \item Select leaders who will implement step 2.
        \item Establish a systematic decision process that uses rights and justice theories. Make this a series of yes or no questions.
        \item Appoint ethics officers who propose, train, monitor, and revise a code and behavior.
        \item Establish corporate governance procedures including
            \begin{itemize}
                \item Majority of the board are independent, outside, and knowledgeable.
                \item The chairman and CEO positions are separate, with the chairman an outside, independent director.
                \item The compensation committee are all independent, outside directors.
                \item The board should retain outside auditors with no conflicts of interest such as providing consulting services to the company.
            \end{itemize}
        \item Show moral courage by supporting managers who make decisions consistent with business ethics, even at the expense of short-term profits.
    \end{enumerate}
\end{flashcard}
\end{document}
