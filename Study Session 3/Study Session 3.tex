\documentclass[../custom,grid]{flashcards}
\usepackage{amsmath}
%\usepackage{enumitem}

\newcommand{\studyArea}{Behavioral Finance}

\def\labelitemii{$\circ$}
\def\labelitemiii{$\diamond$}
\def\labelitemiv{$\cdot$}

\begin{document}
\cardfrontstyle{headings}
\cardfrontfoot{Study Session 3}

\begin{flashcard}[\studyArea]{Micro and Macro Behavioral Finance}
    \begin{itemize}
        \item \textbf{Micro behavioral finance} describes the decision-making process of individuals. It tries to explain why they deviate from traditional finance.
        \item \textbf{Macro behavioral finance} tries to explain how and why markets deviate from the efficiency of traditional finance. 
    \end{itemize}
\end{flashcard}

\begin{flashcard}[\studyArea]{Four Axioms of Rational Decision Makers}
    \begin{itemize}
        \item \textbf{Completeness.} Given a choice between $A$ and $B$, either $A$ or $B$ is preferred, or indifference.
        \item \textbf{Transitivity.} If $A$ is preferred to $B$ is preferred to $C$ then $A$ must be preferred to $C$.
        \item \textbf{Independence.} If $A$ and $B$ are mutually exclusive with $A$ preferred, and $C$ is an additional choice that adds positive utility, then $A + \alpha C$ is preferred to $B + \alpha C$. Here $\alpha C$ is some portion of $C$.
        \item \textbf{Continuity.} If $A$ is preferred to $B$ is preferred to $C$, then there will be a combination of $A$ and $C$ indifferent from $B$.
    \end{itemize}
\end{flashcard}

\begin{flashcard}[\studyArea]{Bayes' Formula}
    \begin{flushleft}
        \begin{align*}
            P(A \vert B) &= \frac{P(B \vert A)}{P(B)} P(A)\\
            \\
            \text{where:}\\
            P(A \vert B) &= \text{probability of $A$ occurring given that $B$ has occurred}\\
            P(B \vert A) &= \text{probability of $B$ occurring given that $A$ has occurred}\\
            P(A) &= \text{probability of $A$ occurring}\\
            P(B) &= \text{probability of $B$ occurring}
        \end{align*}
    \end{flushleft}
\end{flashcard}

\begin{flashcard}[\studyArea]{Risk Aversion in Behavioral Finance}
    \begin{flushleft}
        Traditional finance assumes individuals are risk-averse and prefer greater certainty over less certainty. Behavioral finance uses the following categories.
        \begin{itemize}
            \item \textbf{Risk-averse} have a greater loss of utility for a given loss of wealth than they gain in utility for the same risk in wealth.
            \item \textbf{Risk-neutral} gains or loses the same amount of utility for a given gain or loss of wealth.
            \item \textbf{Risk seeker} gains more utility for a rise in wealth than they lose in an equivalent loss of wealth.
        \end{itemize}
    \end{flushleft}
\end{flashcard}

\begin{flashcard}[\studyArea]{Traditional and Behavioral Finance Utility Functions}
    \begin{itemize}
        \item Traditional finance is based in utility theory with an assumption of diminishing marginal return. This implies
            \begin{itemize}
                \item The risk-averse utility function is concave. As more wealth is added, utility increases at a diminishing rate.
                \item Convex indifference curves due to diminishing marginal rates of substitution.
            \end{itemize}
        \item Behavioral finance observes people who are both risk-seeking and risk-averse. This can lead to complex, double-inflection utility functions.
    \end{itemize}
\end{flashcard}

\begin{flashcard}[\studyArea]{}
    \begin{flushleft}
    \end{flushleft}
\end{flashcard}
\end{document}
